\chapter{Machine Learning-Enhanced Blood-Based Framework}

\section{Introduction}

Precision oncology still confronts challenges for transitioning mathematical models of cancer management from abstraction to practice. Clinical practice has seen only a limited adoption of this, as per \citeinline{metzcar2024review, altrock2015mathematics}, because it has been an essential feature of the nature of things that some parameters have to be patient specific or that they have to be hard and expensive to get. \citeinline{powathil2015systems, anderson2008integrative}, maintained that the employment of conventional methods with parameters derived from population averages is not capable of including the great inter patient variability which is the main characteristic of cancer biology and the response to treatments.

Different types of breast cancer tissues make it difficult to have a certain kind of model. As stated in the references \citeinline{ligero2025artificial, altashi2023machine}, the different molecular profiles, immune capabilities, metabolic states, and resistance mechanisms of patients are the main factors that determine the progression of the disease and the effectiveness of the treatment. \citeinline{henry2012cancer, hori2017model}, showed that population-averaged parameters give rise to models that describe on average but that are not able to predict reliably in the case of individual patients.

Blood-based biomarkers provide an attractive avenue for personalization. \citeinline{duffy2010ca, harris2007american}, demonstrated that modern clinical laboratories routinely measure biomarkers that reflect the tumor burden, proliferation rate, immune function, metabolic state, and resistance to treatment. \citeinline{alix2012circulating, cristofanilli2004circulating}, demonstrated that all of these measurements provide a real-time physiological snapshot that can be repeated with minimal invasiveness and moderate cost. The hypothesis formulated in the present work is that a detailed panel of blood biomarkers contains enough information to derive all parameters needed in mathematical cancer modeling.

Machine learning developments in ensemble learning, gradient boosting algorithms, and neural networks all exhibit powerful capabilities for modeling nonlinear biological relationships \citeinline{kourou2015machine}. The algorithms \citeinline{libbrecht2015machine, caruana2006empirical} can identify complex patterns linking measurable blood values to mathematical parameters that express growth rates, immune interactions, and treatment sensitivities.

This chapter establishes the basis for personalizing optimization of breast cancer treatment, where mathematics, biomarker science and machine learning converge. We have constructed a system of 15 differential equations that capture cancer progression, immune response, treatment effects and resistance evolution in relation to each other. All 37 parameters have been derived from 47 blood biomarkers; eight machine learning algorithms are evaluated for optimal parameter estimation.

Principal innovations include: (1) a systematic methodology by which all model parameters are derived directly from routine blood measurements; (2) extensive evaluation of the machine learning methods; (3) stringency of the stability analysis; (4) strategies to minimize cost; and (5) importance analysis of features.

\section{Methodology}

\subsection{Biomarker Panel Design and Composition}

The design of a biomarker panel begins with the evaluation of a biomarker through all fronts in relation to cancer biology and thereby supports personalized mathematical modeling. Accordingly, we set up a biomarker panel composed of 47 biomarkers ranked into five measurable categories, each dealing with specific components of the mathematical model. The panel consists of a selected mixture that maximizes biological information content while remaining clinically feasible with respect to stipulated laboratory measurements. The reference range for each biomarker is based on clinical laboratory standards and published medical literature. This has been done to enhance the clinical interpretability and applicability of the biomarkers across different healthcare streams, as stated in the paper by  \citeinline{sharma2010chromatin}.

The panel design philosophy revolves around one main point: biologically complete rather than marker redundant. Each biomarker was selected based on established clinical utility, reliable measurement, and specific relevance to estimating mathematical parameters. The five functional groups systematically address the major biological processes represented in our mathematical framework: tumor growth and burden, immune system function and regulation, treatment resistance mechanisms, metabolic state and microenvironment, and organ function affecting drug metabolism and clearance. Such a categorical organization enhances systematic parameter derivation and allows tiered panel optimization along clinical and resource lines.

\subsubsection{Tumor Markers}

Tumor markers include the six biomarkers quantifying cancer cell presence, proliferation activity, tumor burden, and the metastatic potential. These markers are primarily used to estimate the carrying capacity, growth rates, and metastatic parameters of our mathematical framework. 

CA 15-3 is the main breast cancer marker, measuring the mucin 1 glycoprotein overexpressed on adenocarcinoma cells, and had 100\% selection frequency across all machine-learning models. \citeinline{duffy2010ca, gradishar2020breast} have shown normal CA 15-3 values to be below 25 U/mL, elevated from 25-30 U/mL, and high-risk when exceeding 30 U/mL. In contrast, CA 27-29 provides complementary information about tumor burden via another mucin 1-based assay with an alternate epitope specificity, normal value below 38 U/mL, and high-risk threshold above 50 U/mL.

Carcinoembryonic antigen is general carcinoma markers and important to the calculations of the metastatic potential and represents a cell adhesion molecule correlated with invasive capabilities. CEA shows normal value below 3.0 ng/mL and high-risk above 5.0 ng/mL. Thymidine kinase 1 (TK1) is a marker of cell proliferation directly and the main contributor to our proliferation score that we use for calculating growth rates. \citeinline{bitter2020thymidine} showed that TK1, as an S-phase-specific enzyme, directly portrays cell division activity, with normal values below 2.0 U/L and high-risk values above 4.0 U/L.

\citeinline{cristofanilli2004circulating} showed that circulating tumor DNA fraction (ctDNA) offers the opportunity for a real-time assessment of tumor burden, influencing rate of genetic stability and resistance evolution parameters, with normal fractions below 0.5\% and high-risk above 2.0\%. Measuring the amount of ESR1 protein is a necessary step in determining the success of hormone therapy, as it is the receptor that changes the signaling to estrogen in the cell nucleus, with standard concentrations being less than 6.0 ng/mL and a risk area more than 10 ng/mL. The six cancer markers together provide a full assessment which is the starting point for a personal model of tumor dynamics

\subsubsection{Immune Function Markers}

The immune function group comprises twelve biomarkers that reflect not only effector and regulatory immune populations but also cytokine networks and immune checkpoint expressions. These markers serve to parameterize precisely the immune-tumor interplay, immunosuppressive mechanisms, and immune-mediated tumor control in the mathematical framework. CD8+ cytotoxic T cells are the predominant anti-tumor effector population, achieving 100\% selection frequency together with CA 15-3. Normal CD8+ ranges are between 200-1200 cells/$\mu$L, values below 200 indicate immunodeficiency and values exceeding 1200 suggest an active immune status. 

CD4+ helper T cells orchestrate immune responses and contribute to the immune strength score. Normal CD4+ counts are 500-1600 cells/$\mu$L, with values below 500 generally indicating immune dysfunction and values above 1600 indicating immune activation. The CD4:CD8 ratio utilizes additional information regarding immune status and health. Natural killer (NK) cells mediate innate anti-tumor immunity without requiring antigen presentation with normal counts ranging from 90-600 cells. 

Interferon-gamma (IFN-$\gamma$) is a very key marker for the Th1-type immune responses important for anti-tumor immunity with normal IFN-$\gamma$ levels being below 2.0 pg/mL, while any elevated level indicates activation of the immune system. Interleukin-10 (IL-10) is an anti-inflammatory cytokine that mediates immunity and has normal values below 5.0 pg/mL, but elevated values suggest the possibility of tumor-induced immune suppression. Tumor necrosis factor-alpha (TNF-$\alpha$) provides information about pro-inflammatory signaling and has normal values below 8.1 pg/mL. 

Transforming growth factor-beta (TGF-$\beta$) is a powerful immunosuppressive member of cytokines that is elevated in many cancers; normal values go below 2.5 ng/mL. PD-L1 expression on circulating tumor cells (PD-L1 CTC) is indicative of checkpoint activation and potential responses to checkpoint inhibitor therapy, with normal values below 1.0\% and elevated values indicative of immunosuppressive mechanisms.

HLA-DR expression on monocytes reflects antigen presentation capacity, which is essential for activation of the adaptive immune system and ranges between normal values of 40-80\% with low values indicating impaired antigen presentation. Circulating tumor cells (CTC) provide direct evidence for metastatic potential with normal values below 5 cells/7.5mL, while elevated values indicate active metastatic disease \citeinline{cristofanilli2004circulating}. Angiopoietin-2 (Ang-2) is responsible for marking vascular remodeling and angiogenesis, with normal values between 20-500 pg/mL. 

Total lymphocyte count establishes an indicator for the general function of the immune system, with normal values between 500--4000 cells/$\mu$L. Armed with this twelve immune marker ensemble, it becomes possible to characterize the detailed state of the immune system which constitutes the information required for the accurate parameterization of immune-tumor interaction, immunosuppression, and immune-mediated control mechanisms.

\subsubsection{Resistance Markers}

Resistance markers consist of sixteen biomarkers quantifying mechanisms of treatment resistance that include hormonal resistance, multi-drug resistance, pathway-driven resistance, and genetic instability. These markers predict parameters of treatment effectiveness and rates of resistance evolution within our mathematical model. ESR1 mutations indicate resistance mechanisms to hormonal therapy with increasing resistance from wild-type (normal) to heterozygous (emerging resistance) or homozygous/multiple mutations (established resistance). 

PGR protein (progesterone receptor) concentration plays a role in hormone sensitivity, where below 20 ng/mL is considered low, moderate values warrant between 20-100 ng/mL, and high values are established above 100 ng/mL. BRCA mutations disrupt DNA repair capacity and therapeutic sensitivity and can be categorized as wild-type (normal), heterozygous (moderate risk), or homozygous/compound heterozygous (high risk) \citeinline{foulkes2010triple}. 

PIK3CA mutations drive the resistance at pathway levels and gained 91.7\% selection frequency in feature importance analysis. PIK3CA mutations are classified as wild-type, activating mutations (E542K, E545K, H1047R), or multiple/high-impact mutations. TP53 mutations interfere with cell cycle control and apoptosis and are placed in categories based on their mutation types: wild-type, missense mutations, or truncating/dominant-negative mutations.

HER2 mutations and amplifications determine targeted therapy eligibility and effectiveness, classified as normal, moderate amplification/activating mutations, and high amplification/multiple mutations. MDR1 expression quantifies multi-drug resistance pump activity, with low expression (normal), moderate expression (emerging resistance), and high expression (established multi-drug resistance).

MRP1, BCRP, LRP, and ABCG2 serve as other efflux transporters responsible for drug resistance, and are categorized according to their expression. Ki-67 proliferation index indicates information regarding cell cycle progression, with normal values being less than 15\%, moderate 15-30\%, and high above 30\%. Survivin indicates apoptosis inhibition, with normal values being below 0.5 ng/mL, high values between 0.5-2.0 ng/mL, and high-risk values above 2.0 ng/mL. 

Vimentin expression indicates epithelial-mesenchymal transition (EMT), a process associated with metastasis and therapy resistance; N-cadherin is another EMT marker providing complementary information. E-cadherin loss in EMT quantifies loss of epithelial differentiation, with normal expression above 80\%, partial loss between 40-80\%, and severe loss below 40\%. The sixteen resistance markers, therefore, make a comprehensive characterisation of multiple resistance mechanisms for the accurate prediction of treatment effectiveness and resistance evolution dynamics.

\subsubsection{Metabolic Markers}

The metabolic marker category includes eight biomarkers assessing tumor metabolism, systemic metabolic effect, and microenvironmental conditions. These markers give input for the parameterization of microenvironmental conditions, calculations of the metabolic state, and prediction of treatment effectiveness. Glucose level shows information about systemic and tumor metabolism, with normal fasting values of 70-100 mg/dL, pre-diabetes of 100-125 mg/dL, and diabetes above 125 mg/dL. Tumor cells have an enhanced uptake of glucose, which makes this marker relevant for calculating the growth rate. 

Lactate concentration indicates the presence of anaerobic metabolism and formation of an acidic microenvironment, with normal values between 0.5-2.2 mmol/L; elevated values are consistent with enhanced glycolysis. Lactate dehydrogenase (LDH) is a prognostic factor reflecting tissue breakdown and tumor metabolism: normal values are stated as below 250 U/L, and anything higher would indicate an active disease condition. 

Albumin gives information about nutritional status, liver function, and systemic inflammation. It had a selection frequency of 91.7\%. Normal levels of albumin are cited to be 3.5-5.5 g/dL, by which lower levels would mean poor nutritional status or Chronic inflammation. Beta-hydroxybutyrate quantifies ketone body metabolism, with normal fasting values below 0.5 mmol/L, whereas higher values indicate ketosis. 

Blood pH measurement checks for the acid-base balance systemically and might display the effects of tumor acidosis, with normal arterial pH being between 7.35-7.45 and acidosis below 7.35 \citeinline{gatenby2009adaptive}. Folate concentration checks the DNA synthesis and cell proliferation, whereby mutations give normal values of 3-25 ng/mL. Vitamin D levels are one of the factors that affect the immune system and have been linked to cancer results, where normal levels are considered to be 10-80 ng/mL. These eight metabolic markers in combination represent metabolism of the tumor, microenvironmental conditions, and systemic metabolic effect. 

\subsubsection{Organ Function Markers}

Biomarkers related to the liver and kidneys are the focus of this branch since these two organs are the most significant for drug metabolism, the main excretion pathway, and toxicity risk assessment. These markers signal pharmacokinetics and safety features related to the therapy. Creatinine concentration is a measure of kidney function, the reference values for females are 0.6 1.2 mg/dL, and for males, they are 0.7 1.3 mg/dL. The rise in creatinine points to renal clearance that has been reduced and, consequently, a necessary dose adjustment. 

BUN is a great measure of kidney function, with normal levels being within the range of 7 to 20 mg/dL. The BUN: creatinine ratio is a tool that helps to differentiate the specific causes of kidney dysfunction. Alanine aminotransferase or ALT are very sensitive indicators of liver function with values being less than 35 U/L for females and less than 40 U/L for males. Hence, the ALT increase signifies hepatocyte injury and may cause the discontinuation of the drug or treatment replacement.

Complementing liver function determination, aspartate aminotransferase (AST) has normal values of less than 35 U/L in females and less than 40 U/L in males; the AST:ALT ratio helps characterize patterns of liver dysfunction. Total bilirubin assesses liver capability for conjugation and excretion, with normal levels recorded as less than 1.2 mg/dL. Elevation in bilirubin denotes impaired liver function and indirectly interferes with calculations of drug metabolism.

These thus allow the five organ function markers to estimate precisely the pharmacokinetic parameters and safety considerations of the treatment, leading to a patient-specific approach in mathematical prediction of metabolism and clearance, by the drug in each patient. Hence, the 47-biomarker panel accumulates sufficient biological detail to derive all parameters for personalised mathematical cancer modeling.

\subsection{15-Dimensional Mathematical Framework}

Personalized cancer modeling has one foundation, which is a 15-dimensional coupled system of normal ordinary differential equations to describe the dynamics of breast cancer development and immune interactions, effects of therapy, evolution of resistance to therapy, and microenvironmental factors. The overall architecture is meant to deliver biological comprehensiveness within mathematical feasibility for accurate model representation of cancer biology coupled with sufficient practical ease for parameter estimation from blood biomarkers. The system operates within a state space spanned by fifteen separate components, each of which represents a biological compartment and is governed by ordinary differential equations. Each of these equations encodes distinct biological processes and their interactions.

The complete state vector at time $t$ is defined as:
\begin{equation}
\mathbf{Y}(t) = \begin{bmatrix}
N_1(t) \\ N_2(t) \\ I_1(t) \\ I_2(t) \\ P(t) \\ A(t) \\ Q(t) \\ R_1(t) \\ R_2(t) \\ S(t) \\ D(t) \\ D_m(t) \\ G(t) \\ M(t) \\ H(t)
\end{bmatrix}
\end{equation}

where the state variables represent the following biological compartments: $N_1(t)$ would be treatment-sensitive cancer cells susceptible to current therapies; $N_2(t)$ would be partially resistant cancer cells of intermediate sensitivity; $I_1(t)$ quantifies cytotoxic immune cells including CD8+ T cells and NK cells; $I_2(t)$ stands for regulatory immune cells including Tregs and myeloid-derived suppressor cells; $P(t)$ describes metastatic potential and invasion capacity; $A(t)$ measures angiogenesis factors and vascular density; $Q(t)$ indicates quiescent (dormant) cancer cells which are not actively dividing; $R_1(t)$ represents cells with hormone receptor-based resistance; $R_2(t)$ quantifies cells with multi-drug resistance mechanisms; $S(t)$ denotes senescent cells due to treatment-induced damage; $D(t)$ measures active drug concentration in the system; $D_m(t)$ describes the concentration of metabolized drug; $G(t)$ measures genetic stability; $M(t)$ describes metabolic state; $H(t)$ measures hypoxia level in tumors.

The primary study of this framework is the derivation of all 37 core model parameters based on the 47-biomarker blood panel, and thus, truly personalized mathematical modeling. In contrast to classical methods that rely on population-average parameters, the biomarker signature of each patient gives rise to a unique parameter set representative of the individual's cancer biology, immune status, metabolic condition, and treatment susceptibility.

\subsubsection{Core Differential Equation System}

The time evolution of a state variable is described by a differential equation, the right-hand side of which encodes specific biological processes. The complete system is presented below for which the detailed biological interpretation is given for each compartment.

\textbf{Sensitive Cancer Cell Dynamics ($N_1$):}

Sensitive cancer cells have complex dynamics involving logistic growth, killing by immune system activation, effects of treatment upon these cells, and shifts to other cellular states:

\begin{align}
\frac{dN_1}{dt} &= \lambda_1 N_1 \left(1 - \frac{N_{\text{total}}}{K}\right) \frac{(1 + 0.1M)}{(1 + \alpha_{\text{acid}} M)} \nonumber \\
&\quad - \frac{\beta_1 N_1 I_1}{1 + 0.01N_{\text{total}}} - \eta_{\text{treat}} N_1 \nonumber \\
&\quad - \kappa_Q N_1(1 + 0.5H) - \omega_{R1} \eta_{\text{treat}} N_1(2 - G) \nonumber \\
&\quad - \omega_{R2} \eta_{\text{treat}} N_1(2 - G) - \kappa_S \eta_{\text{treat}} N_1(1.3 - 0.3G)
\end{align}

where $N_{\text{total}} = N_1 + N_2 + Q + R_1 + R_2 + S$ is the measure of total burden of tumor across all compartments. The first term includes the intrinsic growth rate $\lambda_1$ estimated from the proliferation biomarkers- TK1, glucose, lactate, survivin, where metabolic factors ($M$) and pH effects ($\alpha_{\text{acid}}$) modulate the logistic growth. This metabolic reprogramming enhances growth through the numerator $(1 + 0.1M)$, while acidotic states partially counteract this through the denominator $(1 + \alpha_{\text{acid}}M)$, which reflects the metabolically influenced complex interaction with proliferation.

The formulation of logistic growth with carrying capacity K provided a classical model in population dynamics and tumor modeling \citeinline{gerlee2013model, benzekry2014classical, murphy2016differences}. The second term represents a rate of cancer cell death induced by the immune system, $\beta_1 .$ This rate was obtained from immune function biomarkers: CD8 + cells, NK cells, and IFN-$\gamma$, with density-dependent saturation modeled via the Holling type-II functional response of $(1 + 0.01N_{\text{total}})^{-1}$ to prevent unrealistic killing rates at very high tumor burdens. The Holling type II functional response has been widely validated in predator-prey and immune-tumor modelling contexts \citeinline{holling1959characteristics, kuznetsov1994nonlinear, depillis2001mathematical}.

Treatment effectiveness $\eta_{\text{treat}}$ is a parameter that integrates personalized effectiveness for different modalities into a single value based on therapy regimens (chemotherapy, endocrine therapy, targeted therapy, immunotherapy). In entering quiescence, the quiescence transition term $\kappa_Q N_1(1 + 0.5H)$ indicates dormancy induction by stress and hypoxia ($H$) as a survival mechanism in the entrance to quiescence.

The resistance-evolution terms $\omega_{R1} \eta_{\text{treat}} N_1(2 - G)$ with $\omega_{R2} \eta_{\text{treat}} N_1(2 - G)$ indicate treatment-induced selection for resistant phenotypes moderated by genetic stability ($G$). Under the assumption that decreased genetic stability accelerates evolution of resistance (with $G$ going to 0), $(2 - G)$ accounts for the biological truth that genomic instability promotes adaptation. The senescence induction term $\kappa_S \eta_{\text{treat}} N_1(1.3 - 0.3G)$ means the permanent growth arrest induced by treatment is also driven by genetic instability.

\textbf{Partially Resistant Cell Dynamics ($N_2$):}

Partially resistant cells exhibit reduced sensitivity to both immune killing and treatment effects:

\begin{align}
\frac{dN_2}{dt} &= \lambda_2 N_2 \left(1 - \frac{N_{\text{total}}}{K}\right) \frac{(1 + 0.1M)}{(1 + \alpha_{\text{acid}} M)} \nonumber \\
&\quad - \frac{0.5\beta_1 N_2 I_1}{1 + 0.01N_{\text{total}}} - 0.7\eta_{\text{treat}} N_2 \nonumber \\
&\quad - \kappa_Q N_2(1 + 0.5H)
\end{align}

The growth rate $\lambda_2$ typically rages from $60-80\%$$; \lambda_1$, indicative of a fitness cost associated with a partial resistance, and is further modulated through resistance markers (like PIK3CA mutations and ESR1 mutations). The immune killing term has been scaled down to 50\% compared to sensitive cells through the 0.5 coefficient, consistent with the enhanced immune evasion capabilities associated with resistance phenotypes. Treatment effectiveness therefore stands at 70\% of the normal, qualifying as indicative of the intermediate sensitivity seen with partially resistant cells. It does not go to completely resistant states directly, but acts as an intermediary phenotype that can be selected under prolonged use of the drug.

\textbf{Cytotoxic Immune Cell Dynamics ($I_1$):}

The CD8+ T cells, and NK cells all belong to a compartment of cytotoxic immune cells whose collective behavior is an effector population capable of antidecting tumoral immune responses.

\begin{equation}
\frac{dI_1}{dt} = \phi_1 + \frac{\phi_2 N_{\text{total}}}{1 + 0.01N_{\text{total}}} - \frac{\beta_2 I_1 I_2}{1 + I_1} - \delta_I I_1(1 + 0.2H) + 0.1\eta_I u_I I_1
\end{equation}

Constitutive immune production $\phi_1$ is related to the recruitment and continuous production of cytotoxic immune cells according to CD4+ T cell counts associated with immune activation markers (IFN-$\gamma$) and lymphocyte counts. Tumor-induced immune activation is $\phi_2$ and depends on the tumor burdens against the adaptive immune recruitment in the sense of Michaelis-Menten saturation kinetics, which models recruitment in more realistic ways at very high tumor burdens. \citeinline{segel1989quasi}.

Therefore, the regulatory immune suppression term $\beta_2 I_1 I_2 / (1 + I_1)$ saturation effects to prevent complete immune depletion because, from biological point of view, certain baseline immune functions exist even in conditions of very strong immunosuppression. The interaction coefficient $\beta_2$ is calculated on behalf of the immunosuppressive markers (IL-10, TGF-$\beta$, PD-L1 expression on CTCs).

Hypoxia promotes an increased immune cell death rate $\delta_I$ by the factor of $(1 + 0.2H)$, which captures the reduced survival of immune cells in this inhospitable tumor microenvironment. The last term means enhancement of immunotherapy: $\eta_I$ is a measure of the ability of the immune checkpoint inhibitors or other immunomodulatory interventions (derived from PD-L1 expression, counts of immune cells, and previous treatment response); $u_I$ is the binary status of immunotherapy administration (0 or 1); and the coefficient 0.1 is the multiplicative enhancement effect.

\textbf{Regulatory Immune Cell Dynamics ($I_2$):}

In the process of creating an immunosuppressive microenvironment, the regulatory immune cells are regulatory T cells (Tregs) and myeloid-derived suppressor cells (MDSCs).

\begin{equation}
\frac{dI_2}{dt} = \frac{\phi_3 N_{\text{total}}}{1 + 0.01N_{\text{total}}} - \delta_I I_2(1 + 0.1H) - 0.1\eta_I u_I I_2
\end{equation}

The regulatory cells $\phi_3$ work in proportionality to tumor burden and, in fact, reflect the tumor's active recruitment of immunosuppressive cells by using cytokines and chemokines (IL-10, TGF-$\beta$, CCL22). The saturation kinetics prevent unbounded expansion of regulatory cells. These cells have death kinetics similar to cytotoxic immune cells via the $\delta_I$ term, slightly enhanced by hypoxia. The last term refers to the depletion of the regulatory cells due to immunotherapeutic measures, notably, checkpoint inhibitors that may reduce Treg function and numbers.

\textbf{Metastatic Potential Evolution ($P$):}

Metastatic ability incorporates several biological indicators toward its process of metastasis:

\begin{equation}
\frac{dP}{dt} = \gamma N_{\text{total}}(1 + 0.5H)(1 + 0.3M) - \delta_P P
\end{equation}

The quantity of metastatic seeding rate $\gamma$ may be inferred from CTC counts, ctDNA fraction, and metastatic biomarkers (epithelial-mesenchymal transition markers like vimentin, N-cadherin; invasion markers like MMP levels where available). Hypoxia enhances metastatic potential by a factor $(1 + 0.5H)$, reflecting the biological observation that hypoxic stress enhances the invasive and metastatic phenotypes through HIF-1$\alpha$ signaling. The metabolic reprogramming increases another factor $(1 + 0.3M)$ of the metastasis, as such altered metabolism meets the energetic demand for invasion and colonization. In very natural ways, $\delta_P$ denotes the rate of degradation, which indicates the clearance of circulating tumor cells and regression of micrometastatic foci through immune surveillance and apoptosis.

\textbf{Angiogenesis Factor Dynamics ($A$):}

Angiogeneic factors (primarily VEGF, plus Angiopoietin-2 if assayed) promote the development of the vasculature beyond microscopic size, needed for the tumor's growth:

\begin{equation}
\frac{dA}{dt} = \frac{\alpha_A N_{\text{total}}(1 + H)}{1 + 0.01N_{\text{total}}} - \delta_A A
\end{equation}

Implies that with increasing tumor burden, angiogenesis induction $\alpha_A$ is enhanced significantly through hypoxia with the factor $(1 + H)$ which translates well into the well-established hypoxia-inducible factor (HIF) pathway upregulating the expression of VEGF under hypoxic conditions. The saturation kinetics also represent how angiogenic factor production does not increase proportionally with tumor size limits. Angiopoietin-2 biomarker availability may modify $\alpha_A$. Spontaneous clearance occurs at rate $\delta_A$ via proteolytic degradation and binding to circulating receptors.

\textbf{Quiescent Cell Dynamics ($Q$):}

Cells in a quiescent state are dormant populations of cancer that are not proliferating at any point in time, which makes them also often resistant to treatment due to low- or metabolic inactivity:

\begin{equation}
\frac{dQ}{dt} = \kappa_Q(N_1 + N_2)(1 + 0.5H) - \frac{\lambda_Q Q(1 + 0.2A)}{1 + 0.5H}
\end{equation}

Quiescence entry rate $\kappa_Q$ encompasses the transition of proliferating cells to dormancy that is greatly aided by hypoxic stress through the factor $(1 + 0.5H)$. This represents a biological strategy of entering quiescence as a safeguard from nutrient and hypoxia deprivation. The reactivation rate $\lambda_Q$ is promoted by angiogenic factors through $(1 + 0.2A)$ when increased vascularization has furnished the nutrients and oxygen necessary for their resumption of proliferation, whilst inhibitory to this process under conditions of sustained hypoxia through the denominator $(1 + 0.5H)$, thus creating a dynamic balance for the cells between entry and exit from quiescence.

\textbf{Hormone-Resistant Cell Dynamics ($R_1$):}

The hormone-resistant cells have emerged from the hormone-sensitive population under pressure from endocrine therapy and possess selective resistance to agents targeting the estrogen receptor:

\begin{equation}
\frac{dR_1}{dt} = \omega_{R1} \eta_E u_E N_1(2 - G) + \lambda_{R1} R_1 \left(1 - \frac{N_{\text{total}}}{K}\right) - \frac{\rho_1 \beta_1 R_1 I_1}{1 + 0.01N_{\text{total}}}
\end{equation}

$\omega_{R1}$ is the rate of resistance evolution that comes about from mutations at ESR1, the status of PGR, and the markers of genetic instability (ctDNA fraction, mutation burden). The evolution takes shape under the pressure of an endocrine treatment, which is related to $\eta_E u_E$, and which has also been sped up by genetic instability through the $(2-G)$. These malignant cells reproduce with a growth rate $\lambda_{R1}$, which can differ from $\lambda_1$, owing to different mechanisms of proliferation that rely less on estrogen. They present partial immune sensitivity with the factor $\rho_1 \in [0.6, 0.9]$, which can be taken as a reduced, but not completely abolished immune recognition of this type of cells in comparison with fully sensitive ones.

\textbf{Multi-Drug Resistant Cell Dynamics ($R_2$):}

This is what describes the most advanced phenomenon of multi-drug resistance cells that have broad resistance mechanisms of treatment such as efflux pumps overexpression as well as metabolic adaptation:

\begin{equation}
\frac{dR_2}{dt} = \omega_{R2} \eta_C u_C N_1(2 - G) + \lambda_{R2} R_2 \left(1 - \frac{N_{\text{total}}}{K}\right) - \frac{\rho_2 \beta_1 R_2 I_1}{1 + 0.01N_{\text{total}}}
\end{equation}

The rate of evolution, denoted by $\omega_{R2}$, is affected not only by MDR1 (ABCB1) expression but also by the presence and/or coexpression of other ATP-binding cassette transporters: MRP1, BCRP, LRP, and ABCG2. In an MRP1, BCRP, and LRP-dependent way, the evolution-resistance mechanism is brought about in the context of a multi-drug resistance score. The above resistance development is induced by chemotherapy pressure through $\eta_C u_C$, and it is further endorsed by genetic instability. The cells showed the highest resistance to immune killing with factor $\rho_2 \in [0.3, 0.6]$. The contribution of reduced antigen presentation and increased immune evasion are reflected as these cells have two opposing characteristics of strong resistance to immune killing. Growth speed $\lambda_{R2}$ could be strongly decreased due to the fitness costs associated with multiple mechanisms of resistance.

\textbf{Senescent Cell Dynamics ($S$):}

Senescent Cells accumulate DNA damage and at the same time induce permanent growth arrest.

\begin{equation}
\frac{dS}{dt} = \kappa_S \eta_{\text{treat}} N_1(1.3 - 0.3G) - \delta_S S
\end{equation}

The senescence induction rate, $\kappa_S$, denotes cells treated with drugs that were either in a permanent growth arrest state or under cytotoxic apopototic stimulus. In the expression $(1.3 - 0.3G)$, we see that senescence induction increases with an increase in genetic instability, as cells with somewhat incapacitated mechanisms for DNA repair are more likely to enter senescence in response to some form of genotoxic stress. The clearance rate, $\delta_S$, reflects the mechanisms of immunity, particularly by NK cells and macrophages, or through secondary necrosis. These senescent cells could exert pro-tumorgenic effects through SASP, although this effect is not modelled per se in the present framework.

\textbf{Active Drug Concentration ($D$):}

Active drug concentration follows standard pharmacokinetic principles with first-order elimination:

\begin{equation}
\frac{dD}{dt} = \text{dose\_rate}(t) - k_{\text{el}} D - k_{\text{metabolism}} D
\end{equation}

The function for dose rate $\text{dose\_rate}(t)$ models the treatment scheme for a given clinical case, under which bolus chemotherapy is given by discrete drug administration events or continuous infusion for hormone therapy. The elimination rate constant $k_{\text{el}}$ has been derived from renal function markers (creatinine, calculated eGFR, BUN), since renal clearance is relevant to most cancer therapeutics. The metabolism rate constant $k_{\text{metabolism}}$ depends on the liver function markers (ALT, AST, bilirubin and albumin) which reflect hepatic cytochrome P450 and conjugation path activity.

\textbf{Metabolized Drug Concentration ($D_m$):}

Drug concentration, or concentration of a metabolized form, in the plasma is presumed to correspond to a organ's metabolic capacity because that will normally reflect the true extent of hepatic metabolism.

\begin{equation}
\frac{dD_m}{dt} = k_{\text{metabolism}} D - k_{\text{clearance}} D_m
\end{equation}

The compartment receives input from drug metabolism and is cleared by renal and biliary excretion at a rate $k_{\text{clearance}}$, also derived from organ function biomarkers. Generally, metabolites are less active than parent compounds, but some retain their activity or contribute to toxicity; hence, following them is relevant to complete pharmacokinetic modeling.

\textbf{Genetic Stability ($G$):}

Genetic stability decreases with increasing tumor burden and the accompanying mutagenic stress experienced from treatment, while having only partial restoration through limited repair mechanisms:

\begin{equation}
\frac{dG}{dt} = -\mu N_{\text{total}} - \nu \eta_{\text{treat}}(2 - G) + \delta_G(1 - G)
\end{equation}

Derivation of mutation accumulation rate $\mu$ is based on ctDNA fraction, somatic mutation burden (if such measure is available), and genetic instability biomarkers (chromosomal instability markers, BRCA mutation status affecting DNA repair capacity). Increased tumor burden speeds up genetic instability because increased replication stress increases the accumulation of errors. Rate $\nu$ captures treatment-induced mutagenesis, which is itself modulated by the following term: $(2-G)$ because cells with less stability are prone to more treatment-induced mutations. Natural Dn repair restores the baseline restoration ability at the rate of $\delta_G$. The factor of $(1-G)$ makes sure that stability does not exceed 1.0 as a maximum value.

\textbf{Metabolic State ($M$):}

The varying conditions under which they operate are captured by means of a metabolic state variable, which is in line with the Warburg effect and with the different metabolic reprogrammings associated with it.

\begin{equation}
\frac{dM}{dt} = \kappa_M N_{\text{total}}(1 + 0.5H) - \delta_M M
\end{equation}

The metabolic reprogramming rate $\kappa_M$ is derived from glucose, lactate, and LDH levels in the blood, which are the direct indicators of tumor glycolysis in cancer metabolism. Metabolic reprogramming is enhanced under hypoxia through the factor $(1+0.5H)$ due to upregulation of glycolytic enzymes by HIF-1$\alpha$ signaling. The beta-hydroxybutyrate levels (ketone metabolism) inversely modulate $\kappa_M$ whenever present. The natural metabolic normalization occurs at the rate $\delta_M$ through homeostatic systemic mechanisms, although generally slow compared to metabolic activation by the tumor.

\textbf{Hypoxia Level ($H$):}

Hypoxia develops when tumor burden exceeds vascular oxygen supply capacity:

\begin{equation}
\frac{dH}{dt} = \kappa_H \max\left(0, \frac{N_{\text{total}}}{K} - 0.5\right) - \alpha_A A H - \delta_H H
\end{equation}

Hypoxia induction occurs only beyond a certain tumor burden (>50\% of the carrying capacity in this formulation) beyond which the oxygen demand exceeds the vascular delivery capacity. The maximum function prevents hypoxia from going negative for low tumor burdens. The term $\alpha_A A H$ describes how angiogenesis reduces hypoxia, as increased vascular density directly improves the oxygen delivery capability proportional to vessel density ($A$) and current hypoxia level ($H$). Natural oxygenation also provides another opportunity for the clearance process where it is defined by the rate of $\delta_H$, which reflects diffusion from well-perfused regions and improvements in the ability of the systemic to deliver oxygen.

An entire fifteen dimensional system of this kind manages to represent all the necessary biological processes that underlie breast cancer progression and is, at the same time, mathematically manageable for parameter estimation and clinical prediction, thus making it a crucial move in this field. The fundamental of personalized modeling lies in the fact that all 37 parameters are systematically derived from the 47 biomarker blood panel, thereby standard laboratory tests being turned into patient specific mathematical models that predict treatment response and guide therapeutic decision making.

\subsection{Parameter Derivation from Blood Biomarkers}

Adjusting the parameters of mathematical models that represent tumor behavior is the methodological innovation for personalizing cancer models. Such a platform associates 47 blood biomarkers that are routinely measured in the clinical practice with 37 mathematical parameters based on the established relationships. The resulting method uses four different ways to model: direct scaling for parameters with biomarker relationships that are defined by simple proportions, composite scoring for complicated processes that need integration of several markers, piecewise functions for threshold dependent behaviors, and bounded transformations to make sure that parameter values remain within a biologically realistic domain

\subsubsection{Framework Architecture and Workflow}

The parameter calculation pipeline is a structured four stage clinical process, which includes quality assurance and uncertainty quantification. In the first step, biomarker preprocessing involves normalization based on clinical reference ranges; imputation of missing values with the median value that is specific to the category; outlier detection, replacement within 3 sigma bounds; and the quality flagging of those measurements that are recognized as clinically implausible. Step 2 calculates composite scores that combine several biomarkers into biologically meaningful indices. Step 3 obtains parameters based on the established formulae that depend on the biologically dictated constraints of the parameters. The validation in Step 4 is extended by checks of biological constraints, checks on parameter consistency, calculation of confidence scores, and evaluation of uncertainty.

\subsubsection{Composite Score Calculations}

Composite scoring goes hand in hand with parameter derivation as it gathers and epitomizes the complex biological processes that are reflected by the integration of multiple biomarkers.

The \textbf{tumor burden score} integrates multiple tumor markers:
\begin{equation}
s_{\text{tumor}} = \frac{1}{5}\left(\frac{\text{CA 15-3}}{31.3} + \frac{\text{CA 27-29}}{38} + \frac{\text{CEA}}{3.0} + \frac{\text{CTC}}{5} + \frac{\text{ctDNA}}{1.0}\right)
\end{equation}

The \textbf{proliferation score} reflects cell division activity:
\begin{equation}
s_{\text{prolif}} = \frac{1}{4}\left(\frac{\text{TK1}}{2.0} + \frac{\text{Glucose}}{95} + \frac{\text{Lactate}}{2.2} + \frac{\text{Survivin}}{0.5}\right)
\end{equation}

The \textbf{immune strength score} quantifies anti-tumor immune capacity:
\begin{align}
s_{\text{immune}} = &\ 0.4\left(\frac{\text{CD8}}{700}\right) + 0.3\left(\frac{\text{CD4}}{1050}\right) \nonumber \\
&+ 0.2\left(\frac{\text{NK}}{345}\right) + 0.1\left(\frac{\text{IFN-}\gamma}{2.0}\right)
\end{align}

The \textbf{immunosuppression score}:
\begin{equation}
s_{\text{suppress}} = \frac{1}{3}\left(\frac{\text{IL-10}}{5.0} + \frac{\text{TGF-}\beta}{2.5} + \frac{\text{PD-L1}}{1.0}\right)
\end{equation}

The \textbf{genetic stability score} (0 to 1):
\begin{align}
G = \max\bigg(&0.1, \min\bigg(1.0, 1 - 0.3\left(\frac{\text{ctDNA}}{1.0}\right) \nonumber \\
&- 0.2\left(\frac{\text{PIK3CA}}{10}\right) - 0.2\left(\frac{\text{TP53}}{10}\right)\bigg)\bigg)
\end{align}

The \textbf{metabolic stress score}:
\begin{equation}
s_{\text{metabolic}} = \frac{1}{3}\left(\frac{\text{Glucose}}{95} + \frac{\text{Lactate}}{2.2} + \frac{\text{LDH}}{250}\right)
\end{equation}

{\color{blue}
The \textbf{stress score} used in immune death rate $\delta_I$, resistance evolution ($\omega_{R1}$, $\omega_{R2}$), and senescence induction $\kappa_S$ is set equal to the metabolic stress score: $s_{\text{stress}} = s_{\text{metabolic}}$.

The \textbf{quiescence score} $s_{\text{quiescence}} \in [0,1]$ reflects nutrient and metabolic stress driving cells into dormancy. It is defined as:
\begin{equation}
s_{\text{quiescence}} = \frac{1}{2}\left(\max\left(0, \frac{100 - \text{Glucose}}{100}\right) + \min\left(1, \frac{\text{Lactate}}{4}\right)\right)
\end{equation}
where low glucose and elevated lactate promote entry into quiescence.
\color{black}}

The \textbf{resistance score type 1} (hormone):
\begin{align}
f_{\text{resist1}} = \max\bigg(&0.1, \min\bigg(2.0, \frac{1}{4}\bigg(\frac{\text{ESR1\_mut}}{8} + \frac{\text{PGR}}{20} \nonumber \\
&+ \frac{\text{PIK3CA}}{5} + \frac{\text{Survivin}}{6}\bigg)\bigg)\bigg)
\end{align}

The \textbf{resistance score type 2} (multi-drug):
\begin{align}
f_{\text{resist2}} = \max\bigg(&0.1, \min\bigg(2.0, \frac{1}{4}\bigg(\frac{\text{HER2\_mut}}{10} \nonumber \\
&+ \frac{\text{MDR1}}{150} + \frac{\text{Survivin}}{6} + \frac{\text{HSP}}{10}\bigg)\bigg)\bigg)
\end{align}

\subsubsection{Growth and Proliferation Parameters}

Sensitive cell growth rate $\lambda_1$ (month$^{-1}$):
\begin{equation}
\lambda_1 = \max(0.01, \min(0.15, 0.04 \times (1 + 1.5 \times s_{\text{prolif}})))
\end{equation}

Partially resistant cell growth rate:
\begin{equation}
\lambda_2 = \max(0.005, \min(0.1, 0.6 \times \lambda_1 \times (1 + 0.5 \times f_{\text{resist1}})))
\end{equation}

Hormone-resistant cell growth:
\begin{equation}
\lambda_{R1} = \max(0.003, \min(0.05, 0.4 \times \lambda_1 \times f_{\text{resist1}}))
\end{equation}

Multi-drug resistant cell growth:
\begin{equation}
\lambda_{R2} = \max(0.001, \min(0.03, 0.25 \times \lambda_1 \times (1 - 0.3 \times f_{\text{resist2}})))
\end{equation}

Carrying capacity $K$ (cells):
\begin{equation}
K = \max(100, \min(15000, s_{\text{tumor}} \times 2000))
\end{equation}

\subsubsection{Immune System Parameters}

Cytotoxic immune killing rate $\beta_1$ (month$^{-1}$):
\begin{equation}
\beta_1 = \max(0.001, \min(0.1, 0.02 \times s_{\text{immune}} \times (1 - s_{\text{suppress}})))
\end{equation}

Regulatory immune suppression rate:
\begin{equation}
\beta_2 = \max(0.01, \min(0.5, 0.05 + 0.15 \times s_{\text{suppress}}))
\end{equation}

Basal cytotoxic immune production:
\begin{equation}
\phi_1 = \max(0.01, \min(0.2, 0.05 + 0.1 \times s_{\text{activation}}))
\end{equation}
where:
{\color{blue}
\begin{equation}
s_{\text{activation}} = \frac{1}{2}\left(\frac{\text{IFN-}\gamma}{5} + \frac{\text{CD4}}{1200}\right)
\end{equation}
where the 47-biomarker panel does not include IL-2; the activation score uses the two available markers (IFN-$\gamma$, CD4) with equal weighting.
\color{black}}

Tumor-induced immune recruitment:
\begin{equation}
\phi_2 = \max(0.005, \min(0.1, 0.01 + 0.03 \times \frac{s_{\text{tumor}}}{2}))
\end{equation}

Regulatory cell recruitment:
\begin{equation}
\phi_3 = \max(0.005, \min(0.15, 0.02 + 0.08 \times \frac{\text{IL-10}}{15}))
\end{equation}

Immune cell death rate:
\begin{equation}
\delta_I = \max(0.02, \min(0.3, 0.05 + 0.1 \times s_{\text{stress}}))
\end{equation}

\subsubsection{Resistance Evolution Parameters}

Hormone resistance evolution rate $\omega_{R1}$ (month$^{-1}$):
\begin{equation}
\omega_{R1} = \max(0.0001, \min(0.01, 0.002 \times s_{\text{genetic}} \times s_{\text{stress}}))
\end{equation}
where:
\begin{align}
s_{\text{genetic}} = \frac{1}{3}\bigg(&\frac{\text{ctDNA}}{1.0} + \frac{\text{PIK3CA}}{10} + \frac{\text{TP53}}{10}\bigg)
\end{align}

Multi-drug resistance evolution:
\begin{equation}
\omega_{R2} = \max(0.0001, \min(0.008, 0.001 \times s_{\text{genetic}} \times s_{\text{stress}}))
\end{equation}

Genetic instability accumulation:
\begin{equation}
\mu = \max(0.001, \min(0.05, 0.01 \times (1 + 1.5 \times s_{\text{genetic}})))
\end{equation}

\subsubsection{Treatment Effectiveness Parameters}

\textbf{Hormone Therapy Effectiveness} ($\eta_E$):
{\color{blue}
\begin{align}
\eta_E = \max\big(&0.1, \min\big(0.95, f_{\text{receptor}} \times f_{\text{metabolism}} \nonumber \\
&\times f_{\text{resist\_hormone}}\big)\big)
\end{align}
\color{black}}

where:
\begin{equation}
f_{\text{receptor}} = \min\left(1.0, \frac{\text{ESR1\_protein}}{6.0}\right)
\end{equation}

\begin{equation}
f_{\text{metabolism}} = \frac{1}{3}(f_{\text{liver}} + f_{\text{CYP2D6}} + f_{\text{general}})
\end{equation}
{\color{blue}
with $f_{\text{CYP2D6}} = \min(1.0, \text{CYP2D6}/2.0)$ (CYP2D6 activity on a scale where normal is $\approx$1--2), and $f_{\text{general}}$ is the same general health factor used for chemotherapy and immunotherapy:
\begin{equation}
f_{\text{general}} = \frac{1}{2}\left(\frac{\text{Albumin}}{4.0} + \max\left(0.5, 1 - 0.3 \times \frac{|95 - \text{Glucose}|}{95}\right)\right)
\end{equation}
\color{black}}

\begin{align}
f_{\text{resist\_hormone}} = 1 - \min\bigg(&0.9, 0.6 \times \frac{\text{ESR1\_mut}}{8} \nonumber \\
&+ 0.4 \times s_{\text{genetic}}\bigg)
\end{align}

Liver function factor:
\begin{align}
f_{\text{liver}} = \frac{1}{3}\bigg(&\max\left(0.2, \min\left(1.2, \frac{40}{\max(\text{ALT}, 5)}\right)\right) \nonumber \\
&+ \max\left(0.2, \min\left(1.2, \frac{45}{\max(\text{AST}, 8)}\right)\right) \nonumber \\
&+ \max\left(0.5, \min\left(1.5, \frac{1.2}{\max(\text{Bili}, 0.1)}\right)\right)\bigg)
\end{align}

\textbf{Chemotherapy Effectiveness} ($\eta_C$):
{\color{blue}
\begin{align}
\eta_C = \max\big(&0.1, \min\big(0.95, f_{\text{general}} \times f_{\text{organs}} \nonumber \\
&\times (1 - 0.7 \times f_{\text{resist2}})\big)\big)
\end{align}
\color{black}}

where:
\begin{align}
f_{\text{general}} = \frac{1}{2}\bigg(&\frac{\text{Albumin}}{4.0} \nonumber \\
&+ \max\left(0.5, 1 - 0.3 \times \frac{|95 - \text{Glucose}|}{95}\right)\bigg)
\end{align}

\begin{equation}
f_{\text{organs}} = \frac{1}{2}(f_{\text{liver}} + f_{\text{kidney}})
\end{equation}

Kidney function factor:
\begin{align}
f_{\text{kidney}} = \frac{1}{2}\bigg(&\max\left(0.3, \min\left(1.3, \frac{1.2}{\max(\text{Creat}, 0.5)}\right)\right) \nonumber \\
&+ \max\left(0.3, \min\left(1.3, \frac{20}{\max(\text{BUN}, 5)}\right)\right)\bigg)
\end{align}

\textbf{HER2-Targeted Therapy} ($\eta_H$):
{\color{blue}
\begin{align}
\eta_H = \max\big(&0.1, \min\big(0.95, f_{\text{HER2}} \times f_{\text{organs}} \nonumber \\
&\times (1 - 0.5 \times f_{\text{resist2}})\big)\big)
\end{align}
\color{black}}

where:
\begin{align}
f_{\text{HER2}} = &\min\left(1.0, \frac{\text{HER2\_circ}}{5.0}\right) \nonumber \\
&\times \left(1 - 0.6 \times \frac{\text{HER2\_mut}}{10}\right)
\end{align}

\textbf{Immunotherapy Effectiveness} ($\eta_I$):
{\color{blue}
\begin{align}
\eta_I = \max\big(&0.1, \min\big(0.95, f_{\text{PDL1}} \times f_{\text{immune\_ctx}} \nonumber \\
&\times f_{\text{general}}\big)\big)
\end{align}
\color{black}}

where:
\begin{equation}
f_{\text{PDL1}} = \min\left(1.0, \frac{\text{PD-L1\_CTC}}{3.0}\right)
\end{equation}

{\color{blue}
\begin{align}
f_{\text{immune\_ctx}} = \frac{1}{4}\bigg(&\frac{\text{CD8}}{700} + \frac{\text{CD4}}{1050} \nonumber \\
&+ \frac{\text{IFN-}\gamma}{2.0} + \max\left(0, 1 - \frac{\text{IL-10}}{15}\right)\bigg)
\end{align}
ensuring the IL-10 term is non-negative for elevated IL-10.
\color{black}}

\subsubsection{Pharmacokinetic Parameters}

Drug elimination rate $k_{\text{el}}$ (day$^{-1}$):
\begin{equation}
k_{\text{el}} = \max(0.05, \min(0.3, \frac{0.1}{f_{\text{clearance}}}))
\end{equation}

where:
\begin{equation}
f_{\text{clearance}} = f_{\text{liver}} \times f_{\text{kidney}}
\end{equation}

Drug metabolism rate (day$^{-1}$):
\begin{equation}
k_{\text{metabolism}} = \max(0.02, \min(0.2, 0.05 \times f_{\text{liver}}))
\end{equation}

Metabolite clearance (day$^{-1}$):
\begin{equation}
k_{\text{clearance}} = \max(0.1, \min(0.5, 0.2 \times f_{\text{clearance}}))
\end{equation}

\subsubsection{Microenvironmental Parameters}

Angiogenesis induction $\alpha_A$ (day$^{-1}$):
\begin{align}
\alpha_A = \max\bigg(&0.001, \min\bigg(0.1, 0.02 \times \left(1 + \frac{\text{VEGF}}{400}\right) \nonumber \\
&\times \left(1 + \frac{\text{Ang-2}}{3000}\right)\bigg)\bigg)
\end{align}

Angiogenesis degradation:
\begin{equation}
\delta_A = \max(0.05, \min(0.2, 0.1 \times f_{\text{clearance}}))
\end{equation}

Quiescence entry rate:
\begin{equation}
\kappa_Q = \max(0.001, \min(0.05, 0.005 + 0.02 \times s_{\text{quiescence}}))
\end{equation}

Quiescence exit rate:
\begin{equation}
\lambda_Q = \max(0.0005, \min(0.02, 0.002 + 0.01 \times (1 - s_{\text{quiescence}})))
\end{equation}

Senescence induction:
\begin{equation}
\kappa_S = \max(0.001, \min(0.04, 0.002 + 0.01 \times s_{\text{stress}}))
\end{equation}

Senescent cell clearance:
\begin{equation}
\delta_S = \max(0.02, \min(0.1, 0.05 \times s_{\text{immune}}))
\end{equation}

Metastatic seeding rate:
\begin{equation}
\gamma = \max(0.0001, \min(0.01, 0.002 \times f_{\text{metastatic}}))
\end{equation}

where:
\begin{equation}
f_{\text{metastatic}} = \frac{1}{3}\left(\frac{\text{CTC}}{20} + f_{\text{EMT}} + \frac{\text{Exosomes}}{100}\right)
\end{equation}

\begin{equation}
f_{\text{EMT}} = \max\left(0, \frac{5 - \text{miR-200}}{5}\right)
\end{equation}

Metastatic clearance:
\begin{equation}
\delta_P = \max(0.02, \min(0.1, 0.05 + 0.03 \times s_{\text{immune}}))
\end{equation}

{\color{blue}
\subsubsection{ODE Supplementary: Acidosis Modulation}

The growth terms in the $N_1$ and $N_2$ differential equations include the factor $(1 + 0.1M)/(1 + \alpha_{\text{acid}} M)$, where acidosis modulates growth. The parameter $\alpha_{\text{acid}}$ is derived from blood pH (normal arterial pH 7.35--7.45; acidosis below 7.35):
\begin{equation}
\alpha_{\text{acid}} = \max\left(0.01, \min\left(0.5, 2 \times (7.4 - \text{Blood pH})\right)\right)
\end{equation}
Lower pH increases $\alpha_{\text{acid}}$, reducing the effective growth rate in acidotic microenvironments. This parameter is used in the ODEs but is not counted among the 37 core parameters.
\color{black}}

\subsubsection{Quality Control and Validation}

Biological constraint validation:
\begin{align}
\text{Hierarchy:} &\quad \lambda_1 > \lambda_2 > \lambda_{R1} > \lambda_{R2} \\
\text{Effectiveness:} &\quad 0.1 \leq \eta_i \leq 0.95 \\
\text{Immune:} &\quad \beta_1 > 0.001 \\
\text{Resistance:} &\quad \omega_{R1}, \omega_{R2} < 0.01
\end{align}

Model confidence:
\begin{align}
\text{confidence} = \frac{1}{4}\big(&c_{\text{complete}} + c_{\text{consistent}} \nonumber \\
&+ c_{\text{valid}} + c_{\text{stable}}\big)
\end{align}

Such a unified system changes regular blood tests into the personalized model parameters while also measuring confidence, thus enabling accurate patient specific cancer progression modeling that makes sense clinically and is biologically valid.

\subsection{Machine Learning Implementation}

Systematic biomarker mapping to parameters methodologically provides for an initial estimation grounded in biological relationships. However, the analysis of complex non linear relationships allows one to argue for the use of non linear methods to achieve improved predictive accuracy when trained on historical data. We set up a clearly defined machine learning framework in which we challenged eight cutting edge algorithms to work against the 18 main parameter models and measured their performance. By means of this systematic assessment, we were finally able to obtain empirical evidence that could identify the most appropriate algorithms for use and gauge the decrease in performance that could have been caused if machine learning augmentation had not been applied.

\subsubsection{Synthetic Dataset Generation}

High-quality synthetic datasets with realistic biomarker profiles and corresponding ground-truth parameters were needed for training and validation purposes of the machine-learning models developed for various cancer applications. Based on biologically plausible distributions and validated correlation structures, 5,000 synthetic cancer patients were generated. This allows thorough evaluation of models without the heavy time constraints and expenses that are required for the collection of real patient data.

The artificial dataset generation considered three critical biological components that control the variation of the biomarkers. The tumor base factor had an exponential distribution with a scale parameter set to 1.0 and represented the overall tumor burden and aggressiveness, driving all the elevations of tumor markers. The immune base factor followed a gamma distribution with parameters shape 2 and scale 1, being inversely related to tumor burden with the relationship $\text{immune\_base} = \gamma(2,1) / (1 + 0.3 \cdot \text{tumor\_base})$, which showed that immune function becomes weak with increased tumor burden. The genetic instability base factor had an exponential distribution with scale 1.0, forming the resistance marker amplification.

Each of the 47 biomarkers was generated in terms of particular mathematical relationships that reflected the biological constraints. Tumor markers from the tumor base factor conventionally were CA 15-3, followed by a lognormal distribution scaled by tumor burden; CA 27-29 was associated with CA 15-3 multiplied by uniform(1.2, 1.8) plus Gaussian noise; and CEA derived from CA 15-3 was scaled for uniform(0.08, 0.15). Immune markers obtained from immune base factor: normal distribution scaled by immune strength, CD4+ cells associated with CD8+ through uniform(1.2, 1.8) scaling, and cytokines represented immune activation states.

These realistic correlations were given for the relevant biomarkers so that the resulting picture would ultimately seem biologically plausible. Strong correlations (0.8-0.9), typically connected between closely related pairs, were present for CA 15-3 and CA 27-29, as well as liver function markers ALT and AST. These immunologic metrics also showed correlation with moderate (0.5-0.7) linkage between different immune cell populations such as CD8+ and CD4+ T cells. The tumor base factor caused strong correlations (0.7-0.9) in all the tumor markers, echoing biological reality that multiple marker elevations are driven by tumor burden.

Patient metadata like age, disease stage, and previous treatments were generated from realistic distributions. Age was modeled according to a normal distribution that had a mean of 58 years and a standard deviation of 12 years, which was representative of the current demographics with regard to breast cancer. And disease stage was categorized across stages I-IV with probabilities reflecting population prevalence. Such ultra-robust synthetic data allowed for rigorous training towards machine learning models while still maintaining biological realism.

\subsubsection{Algorithm Selection and Configuration}

We examined eight different machine learning algorithms reflecting varying methodological approaches. The gradient boosting methods XGBoost, LightGBM, and CatBoost build ensembles of decision trees via iterative refinement and have demonstrated outstanding performance in structured data problems as well as in medical applications. The classical ensemble methods for Random Forest and Extra Trees pool their predictions from several decorrelated decision trees.

Support Vector Regression with radial basis function kernels represents a non-linear modeling technique based on statistical learning theory in kernel-based framework. Neural networks structured with 2 hidden layers using 64 and 32 nodes with ReLU activation functions represented the deep-learning techniques capable of learning a complex non-linear relationship. Ridge regression acted as the regularized linear baseline to compare against the simple linear models.

All algorithms were put through careful hyperparameter optimization via grid search with 5-fold cross-validation. Hyperparameters of XGBoost include learning rate (0.01-0.3), max tree depth (3-10), number of estimators (100-1000), and subsample ratio (0.6-1.0). LightGBM was tuned for learning rate (0.01-0.3), number of leaves (20-100), maximum depth (3-10), and feature fraction (0.6-1.0). CatBoost hyperparameter tuning involved learning rate (0.01-0.3), depth (4-10), iterations (100-1000), and L2 regularization (1-10).

Hyperparameters of Random Forests included the number of trees (100-500), maximum depth (10-50), minimum samples split (2-10), and maximum features (from square root to total features). Extra Trees used similar ranges with additional randomization. The SVR optimization was done over C parameter (0.1-100), epsilon (0.01-1.0), and gamma (0.001-1.0). As far as the configuration of the neural network is concerned, it was shaped around learning rate (0.0001-0.01), batch size (32-256), and dropout rate (0.0-0.5). Such a detailed hyperparameter optimization ensured a fair comparison across all the algorithms.

\subsubsection{Training and Evaluation Protocol}

Synthetic patients of 5000 were randomised into training (70\%; 3,500 patients), validation (15\%, 750 patients), and test (15\%, 750 patients) data sets with a fixed, reproducible random seed. The machine learning algorithms trained separately for each of the 18 main parameters for parameter-specific algorithm selection and performance characterization. 

Training used early stopping based on validation performance to minimize overfitting: tree-based methods trained until performance on validation did not improve for 50 consecutive iterations; neural nets trained for 1000 epochs maximum, and early stop if validation showed no improvement for 100 epochs. These two approaches achieve a good balance between enough training and good computational efficiency. 

A variety of complementary metrics assessing different dimensions of prediction quality were employed for the performance evaluation. The coefficient of determination ($R^2$) expresses the percentage of variance in the data that is explained by the predictions, the higher the $R^2$, the better the fit. RMSE expresses how wrong the predictions were, with the advantage of being weighted in favour of larger errors, and gives one a feel for the typical accuracy of the predictions, whereas MAE adds to this information since it is not so sensitive to outliers compared to RMSE.

Stratified 5 fold partitioning in cross validation resulted in very robust performance estimates. It made sure that each fold included representative distributions of biomarker values and parameters. The average and standard deviation of the performance metrics across the folds acted as an indicator of the performance and its variability that could be expected. A thorough evaluation like this is a guarantee for a reliable performance characterization and thus, an algorithm selection with the same level of confidence.

\subsubsection{Ensemble Modeling Strategy}

In conjunction with assessing single algorithms, we incorporated ensemble modeling to take predictions from multiple algorithms and pool them together in the hope that they would become stronger through the complementarity provided to each other. Such ensemble strategies include: simple average of all the algorithms' predictions, weighted average using weights equal to validation $R^2$ scores, and stacked ensembling with ridge regression to learn optimal combination weights. 

The ensemble technique provides in-built uncertainty quantification. The inherent variance of prediction across constituent models would show areas of high variance as indicating parameters where the algorithms disagree, indicating a lower confidence level. On the other hand, areas having low variance are those where predictions are robust with high agreement across methods. This particular method of uncertainty quantification could be effective for clinical decision support, regarding the reliability of the predictions made.

Performance of ensembles was evaluated by using exactly the same metrics that were applied to performance evaluation of an individual algorithm, using similar cross-validation schemes for models. Moreover, computational costs were measured in terms of the time needed for training and inference of each method, thus maintaining the methods' practicality. This detailed account of the framework for the implementation of machine learning is a huge step in the right direction towards that one crucial aspect of blood biomarkers: to have parameter estimation that is as accurate, reliable, and computationally efficient as possible.

\subsection{Stability Analysis Methodology}

Stability analysis using mathematics influences the way the model behaves in a biologically plausible manner when the parameters are derived. Hence, stability related detailed analysis protocols have been established: it is possible to check local linearization stability through eigenvalue analysis of the Jacobian and also consider global behavior via long term numerical simulation. Such a twofold method gives assurance that the individual parameters employed will result in mathematically stable models for clinical prediction.

\subsubsection{Jacobian Matrix Construction}

Local stability analysis essentially implies linear analysis of the model behavior in the vicinity of the equilibrium point. The Jacobian matrix, represented by $\mathbf{J}$, may be considered as a set of partial derivatives which describe how one differential equation changes with respect to each state variable.

\begin{equation}
J_{ij} = \frac{\partial f_i}{\partial Y_j}
\end{equation}

Here, $f_i$ represents the right hand side of the $i$th differential equation and $Y_j$ denotes the $j$th state variable as per the commonly accepted norms of system dynamical analysis \citeinline{strogatz2015nonlinear}: for a 15 dimensional system, a 15 by 15 matrix at equilibria is constructed. 

The Jacobian matrix was computed analytically to ensure that a maximum degree of accuracy was attained and allowed symbolic analysis of the elements. Each element was derived symbolically using computer algebra systems and implemented in numerical code for evaluation at specific parameter sets. This way, it is ensured that analytical derivatives ensure the elimination of all errors that may accrue due to numerical approximations, which in turn not only corrupt the assessment of stability.

\subsubsection{Equilibrium Identification}

Here we shall show the equilibria, which satisfy the equation $d\mathbf{Y}/dt = \mathbf{0}$ indicating steady states where every process is balanced. We covered multiple approaches for equilibria identification to ensure Detailed coverage. Numerical root-finding approaches using a trust-region dogleg method were implemented from different initial conditions that spanned the biologically realistic state space. Through analytical simplifications, equilibria were determined for those subsystems solvable exactly. Biological reasoning further helped to identify those equilibria clinically relevant to tumor control, progressive disease, and treatment-induced equilibria.

For each parameter set, we identified all equilibria within the non-negative orthant (since cell populations less than 0 are non-physical) and calculated the Jacobian at each equilibrium. This ensured all mathematically and biologically relevant steady states were assessed for stability.

\subsubsection{Eigenvalue Spectrum Analysis}

The eigenvalues of the Jacobian matrix at the equilibrium evaluate the stability of such equilibrium. An equilibrium is said to be locally asymptotically stable if all the eigenvalues have negative real parts, indicating that all small perturbations will decay over time. Eigenvalues having positive real parts denote that there is instability and that the perturbation increases exponentially. On the other hand, complex eigenvalues with real parts not equal to zero indicate oscillatory dynamics.

We used the QR algorithm with implicit shifts that yield numerical stability for this eigenvalue problem to compute the full spectrum of eigenvalues for every equilibrium. For every eigenvalue $\lambda = \lambda_r + i\lambda_i$, we recorded real part $\lambda_r$, imaginary part $\lambda_i$, and magnitude $|\lambda|$. Summary statistics included the maximum real part (most destabilizing eigenvalue), trace of the Jacobian (sum of eigenvalues), and its determinant (product of eigenvalues). 

Stability classification made use of strict criteria: for full stability, it's required that all eigenvalues have negative real parts ($\lambda_r < -\epsilon$ where $\epsilon = 10^{-6}$ accounts for numerical precision). Partial stability refers to some stable eigenvalues with at least one misleading union. Most of positive real parts is termed as instability. Where configurations are partially stable, the count of stable eigenvalues represented degree of stability.

\subsubsection{Parameter Set Sampling}

Stability assessment of intricate models should include an evaluation across realistic biological parameters-in-the-space.  We carefully designed 15 different parameter sets that span the configuration space derivable from our panel of biomarkers. Sampling of parameters was done according to stratification of the tumor burden situation states low, medium, and high; corresponding strong, moderate, and weak immune response settings; various possibilities of resistance; and metabolic and organ function states.

From each parameter set, all 37 parameters were derived from the respective biomarker profile using a systematic mapping framework. This way, the analyzed combinations were not only clinically valid, but they were also not just some arbitrary mathematical concoction. Altogether, the 15 parameter sets would signify expected population variability in a clinical population and thus provide a fair representation for stability characterization.

\subsubsection{Long-Term Simulation Validation}

Detailed stability analysis in addition to the long-term numerical simulations done earlier, as much as 500 time units (~1-2 years of disease progression) were performed. The fourth-fifth order Runge-Kutta adaptive step-size solver with relative tolerance $10^{-6}$ and absolute tolerance $10^{-8}$ was used to carry out the integration processes in order to achieve accuracy for the integration of possibly stiff equations.

From biologically realistic initial conditions representing various presentations of the disease in order to simulate the parameters set. We looked for pathological behaviours such as unbounded growth, negative populations, numerical instability, or equilibrium values far from the reasonable. An acceptable solution is characterized by bounded populations consistent with clinical findings, smooth trajectories devoid of erratic oscillations, and equilibria within physiologically plausible limits.

Such a comprehensive stability analysis ensures that patient-specific parameters obtained from blood-based biomarkers indeed produce mathematically stable models with biologically realistic dynamics. The combination of analytical eigenvalue analysis with numerical simulation gives high confidence in the reliability of the models for clinical purposes.

\section{Results}

\subsection{Machine Learning Performance}

A through assessment has been made of eight machine learning models regarding their performance with respect to 18 parameters of the mathematical model based on a synthetic population of 5000 cancer patients. The formed hierarchies were discriminative: tree-based gradient boosting methods were very high on predictive accuracy. In all, 144 individual models were formed through training (8 algorithms × 18 parameters) to enable systematic comparisons across the differing algorithmic paradigms of ensemble methods, kernel-based approaches, neural networks, and regularized linear models. The assessments included exhaustive cross-validation protocols and complementary performance metrics to strengthen their clinical applicability.

\subsubsection{Overall Algorithm Performance}

CatBoost turned out to be the top most-performing algorithm, with a mean $R^2 = 0.996 \pm 0.003$ across parameters and about 3.91 mean absolute and 5.67 root mean squared errors as summarized in Table ~\ref{tab:ml_performance_summary}. The low standard deviation of merely 0.003 reflects high consistency across all parameters of a model and suggests that CatBoost is able to capture all the different biomarker-parameter relationships without requiring parameter or architecture-based special tuning. This would be especially important for cases in which clinical implementation requires performance that is robust across all parameters for deep characterization of patient profiles.

\begin{table}[htbp]
\centering
\caption{Machine Learning Algorithm Performance Summary Across All Parameters}
\label{tab:ml_performance_summary}
\small
\begin{adjustbox}{max width=\textwidth}
\begin{tabular}{@{}lcccccc@{}}
\toprule
\textbf{Algorithm} & \textbf{Mean $R^2$} & \textbf{Std $R^2$} & \textbf{MAE} & \textbf{RMSE} & \textbf{MAPE (\%)} & \textbf{CV Std} \\
\midrule
CatBoost & 0.996 & 0.003 & 3.91 & 5.67 & 2.8 & 0.004 \\
LightGBM & 0.992 & 0.006 & 4.08 & 6.12 & 3.2 & 0.005 \\
XGBoost & 0.985 & 0.014 & 6.47 & 8.45 & 4.5 & 0.018 \\
Random Forest & 0.979 & 0.021 & 4.76 & 7.24 & 3.9 & 0.025 \\
Extra Trees & 0.978 & 0.020 & 6.72 & 8.89 & 5.1 & 0.027 \\
SVR & 0.201 & 0.245 & 95.93 & 124.56 & 68.3 & 0.256 \\
Neural Networks & 0.310 & 0.285 & 87.28 & 115.42 & 62.1 & 0.298 \\
ElasticNet & 0.178 & 0.267 & 98.12 & 132.78 & 71.5 & 0.278 \\
\midrule
\textbf{Ensemble} & \textbf{0.997} & \textbf{0.002} & \textbf{3.54} & \textbf{5.12} & \textbf{2.5} & \textbf{0.003} \\
\bottomrule
\end{tabular}
\end{adjustbox}
\end{table}

LightGBM was very close with $R^2=0.992\pm 0.006$, an MAE of 4.08, and an RMSE of 6.12; XGBoost achieved $R^2=0.985\pm0.014$ with an MAE of 6.47, whereas the RMSE was 8.45. The performance ordering confirms the superiority of different gradient boosting methods for biomarker-parameter mapping with all three algorithms explaining over 98.5%: Random Forest and Extra Trees reflecting conventional ensemble methods without gradient boosting, respectively provided commendable performance with $R^2$ scores of 0.979 and 0.978 but they lagged behind gradient boosting methods in the range of approximately 1-2% explained variance.

The specific performance comparisons are all shown in Figure \ref{fig:ml_performance_comparison}, where the different algorithms can easily be seen being grouped within the different categories of algorithms. Tree-based ensemble methods (CatBoost, LightGBM, XGBoost, Extra Trees, Random Forest), however, achieved excellent predictive power with $R^2$ scores above 0.97, which certainly formed an elite high-performance tier. Traditional linear methods (ElasticNet) and kernel-based methods (SVR) performed really poorly, as did those tree-based ensemble methods, with scores below 0.25 for $R^2$. Such ignorant scores should show basic limitations in capturing biomarker-parameter relationships that are not linear. The neural networks did moderately, as indicated by $R^2=0.310$, but highly variant ($\sigma=0.285$) suggesting overfitting and inconsistent generalization with respect to different parameters.

\begin{figure}[htbp]
\centering
\includegraphics[width=1.0\textwidth]{figures/06_ml_performance_comparison.png}
\caption{Machine Learning Model Performance Comparison showing $R^2$ scores, accuracy heatmaps, mean absolute errors, and score distributions across algorithms.}
\label{fig:ml_performance_comparison}
\end{figure}

The $R^2$ score heatmap in the upper right panel of Figure \ref{fig:ml_performance_comparison} is unaware of the overall performance trends along the parameters where gradient boosting algorithms achieve high accuracies, thus it seems that the algorithms are not particularly interested in the parameter they are predicting. Such a high level of agreement to a large extent reveals the fundamental power of gradient boosting in terms of capturing complex nonlinear interdependencies and automatically handling feature interactions, which happen to be quite consistent with the intrinsic properties of biomarker parameter mappings. Besides that, the heat map shows that such growth parameters generally have slightly better predictability than most treatment effectiveness parameters across all algorithms, which points to different complexities of their underlying biological relationships.

\subsubsection{Parameter-Specific Performance Analysis}

Deeply delving into the prediction performance across varied parametric categories, it appears that one-dimensional structures indicate a major variation in the relationship of the biomarker to the parameters among the parameters considered, thus possibly having a role in model selection along with clinical interpretation. Growth-related parameters have been said to be generally more predictable than the treatment effectiveness parameters, as exhibited consistently across each algorithm and confirmed by both the primary metrics and cross-validation analysis.

Growth and proliferation parameters like carrying capacity ($K$), sensitive cell growth rate ($\lambda_1$), and partially resistant growth rate ($\lambda_2$) delivered near - perfect prediction accuracy with $R^2 = 0.999$-1.000 for all tree-based methods. Such high performance derives from how these parameters are directly connected with tumor burden markers, wherein carrying capacity almost linearly scales with CA 15-3, CA 27-29, and CEA concentrations and how the growth rates strongly monotonic relate to proliferation biomarkers such as TK1, glucose, and survivin. It is the high predictability of growth parameters which provides that additional confidence for accurate patient-specific tumor growth dynamics characterization from routine blood assessment.

Immune production parameters also achieved an R-squared of 1.000 across the tree methods, clearly defined by immune cell numbers and levels of cytokines. Baseline immune production $\phi_1$ maps directly to CD4+ T cell counts and activation markers, while tumor-induced immune recruitment $\phi_2$ shows simple relations to tumor burden scores. This high predictability confirms the biological hypothesis that immune recruitment rates can be inferred from circulating immune cell populations and their activation states identifiable by standard flow cytometry or cytokine assays.

These immune-system interaction parameters had lower but still great predictability, with immune killing rate ($\beta_1$) achieving $R^2 = 0.97$-0.99 across the best algorithms. Compared with the immune production parameters, this slight lack of precision points to the increased complexity in immunological killing dynamics combined within integrated cytotoxic capacity (CD8+ cells, NK cells, IFN-$\gamma$), tumor susceptibility factors, and immunosuppressive dampening (IL-10, TGF-$\beta$, PD-L1). Machine learning has to learn non-linear integration patterns rather than simple scaling relationships because of the composite nature of these relationships. Therefore, both reduced accuracy and performance gap created between the more sophisticated algorithms (CatBoost, LightGBM) and simpler methods (Random Forest, SVR) can be explained.

Resistance evolution parameters ($\omega_{R1}$, $\omega_{R2}$) had an R-squared of 0.94 - 0.98, and their predictability was derived from the genetic instability markers (ctDNA fraction, PIK3CA mutations, TP53 mutations) and treatment-induced mutagenic factors. The fact that their predictions are not the best when compared with those for growth parameters likely indicates the stochasticity inherent in evolution of resistance and limitations on derivation of intratumoral genetic dynamics from blood-based measures. These data were, however, sufficiently accurate to suggest that circulating biomarkers give enough information to model reasonably accurate resistance evolution rates for treatment optimization purposes.

The treatment effectiveness parameters ($\eta_E$, $\eta_C$, $\eta_H$, and $\eta_I$) had the most complicated prediction patterns at the top end of this scale ($R^2 = 0.95$-0.99) and required combining multiple biomarker classes, including receptor status, resistance markers, immune function indicators, and organ function measurements, to be predicted. Among these, the endocrine therapy effectiveness $\eta_E$ reached the best accuracy of $R^2 = 0.995$ due to clearer mechanistic linkage with estrogen receptor expression and ESR1 mutation status. Chemotherapy effectiveness $\eta_C$ reached $R^2 = 0.991$, immunotherapy effectiveness $\eta_I$ achieved $R^2 = 0.989$, and targeted therapy effectiveness $\eta_H$ obtained $R^2 = 0.987$. The higher the hierarchical accuracy in ordering, the greater the degree of reliance on multi-factorial integration as chemotherapy invokes consideration of global health indicators and organ functioning, immunotherapy throws new elements of complexity on the immune contexture, and targeted therapy likewise includes receptor heterogeneity through mutation-driven mechanisms of resistance.

\subsubsection{Cross-Validation and Generalization Analysis}

The detailed cross validation analysis supported the rapid understanding and generalization features of the machine learning models, and the tree based methods were very stable across different validation folds. The cross validated performance of the top algorithms, quite similar to the test set performance, was a strong argument for their real predictive capacity rather than just overfitting to random data partitions or some spurious correlations in the training set. 

The highest cross validation $R^2=0.995\pm0.004$ for CatBoost is a strong indication of very good and consistent algorithmic performance across data folds. The minor drop in $R^2$ from the test set $0.996$ to the cross-validation $0.995$ has to be normal, thereby substantiating the virtually absence of any overfitting. LightGBM proved equally robust with cross-validation $R^2 = 0.993 \pm 0.005$ and XGBoost $R^2 = 0.991 \pm 0.006$, as shown in Table \ref{tab:cv_performance_analysis}.

\begin{table}[htbp]
\centering
\caption{Cross-Validation Performance Analysis Results}
\label{tab:cv_performance_analysis}
\small
\begin{tabular}{@{}lccc@{}}
\toprule
\textbf{Algorithm} & \textbf{Mean CV $R^2$} & \textbf{CV Std Dev} & \textbf{Stability Rank} \\
\midrule
CatBoost & 0.995 & 0.004 & 1 \\
LightGBM & 0.993 & 0.005 & 2 \\
XGBoost & 0.991 & 0.006 & 3 \\
Random Forest & 0.978 & 0.015 & 4 \\
Extra Trees & 0.977 & 0.018 & 5 \\
Neural Networks & 0.305 & 0.298 & 6 \\
SVR & 0.195 & 0.256 & 7 \\
ElasticNet & 0.172 & 0.278 & 8 \\
\bottomrule
\end{tabular}
\end{table}

Gradient boosting methods achieve low cross-validation standard deviations (0.004-0.006), confirming generalization without overfitting. The learning paradigm of these algorithms encompasses basic biomarker-parameter relationships, rather than simple memorization of training examples or exploitation of spurious correlations. They would appear less rigorous by contrast to much more heterogeneous methods of the traditional school: $R^2 = 0.195 \pm 0.256$ for SVR and $R^2 = 0.305 \pm 0.298$ for neural networks, the standard deviations being greater than 25\% of the mean value. These large variations indicate instability and unreliable predictions through different patient populations, hence rendering such methods unsuitable for clinical applications requiring consistent performance. 

Cross-validation score distributions shown in Figure \ref{fig:cv_analysis} show that tree-based methods maintain low variance from fold to fold, with interquartile ranges remaining below 0.02 for CatBoost, LightGBM, and XGBoost. This clustering of values close to the mean indicates that performance is stable irrespective of which patient subset comprises the validation fold, thus providing confidence that these models will generalize to new patient populations. Such consistency across different biological situations represented in various folds is an indication that the learned biomarker-parameter mappings capture some fundamental biological reality, rather than dataset-specific one.

\begin{figure}[htbp]
\centering
\includegraphics[width=1.0\textwidth]{figures/03_cv_results_analysis.png}
\caption{Cross-Validation Results Analysis showing CV score distributions, parameter-wise performance consistency, and stability relationships across algorithms.}
\label{fig:cv_analysis}
\end{figure}

Parameter-specific cross-validation analysis resulted in prediction stability being perceived differently by parameters. Growth parameters were least variable (cross-validation standard deviation of 0.002 to 0.003), indicating great stability of prediction irrespective of dataset partition. Immune parameters were moderately variable (standard deviation of 0.004-0.006), whereas treatment effectiveness parameters slightly higher but still within acceptable limits (standard deviation of 0.006-0.009). This pattern suggests that more direct biomarker relationship parameters show more stable predictions, whereas parameters needing evolutionary integration of multiple biomarkers present with some additional variability that reflects the increased challenge of learning complex non-linear relations given the limitation of training data.

\subsubsection{Ensemble Performance and Uncertainty Quantification}

The ensemble method that averaged predictions from multiple algorithms produced a modest but consistent improvement over its individual counterpart while offering a measure of uncertainty with respect to variance in predictions. The ensemble averaged the predictions of the three algorithms with the best performance (CatBoost, LightGBM, XGBoost) with weights proportional to their respective performances with regard to validation $R^2$.

An ensemble with mean $R^2 = 0.997 \pm 0.002$, MAE of 3.54, and RMSE of 5.12 has improved upon the performance of CatBoost alone by approximately 0.1\% in $R^2$ and 9.5\% in MAE. These may seem like trivial improvements in absolute terms and may well not compensate for all the efforts, but these gains have shown consistent utility across almost all parameters (see Table \ref{tab:ensemble_performance}).

\begin{table}[htbp]
\centering
\caption{Ensemble Model Performance Improvement by Parameter Category}
\label{tab:ensemble_performance}
\small
\begin{tabular}{@{}lcccc@{}}
\toprule
\textbf{Category} & \textbf{Single Best $R^2$} & \textbf{Ensemble $R^2$} & \textbf{Improvement} & \textbf{Uncertainty (SD)} \\
\midrule
Growth Parameters & 0.9995 & 1.0000 & 0.05\% & 2.1 \\
Immune Parameters & 0.9940 & 0.9965 & 0.25\% & 3.8 \\
Resistance Parameters & 0.9650 & 0.9750 & 1.04\% & 5.4 \\
Treatment Effectiveness & 0.9880 & 0.9950 & 0.71\% & 4.2 \\
\bottomrule
\end{tabular}
\end{table}

Ensemble performance improvements were very dependent on the parameters with growth parameters seeing very little enhancement (0.0-0.1%) as these parameters are already at almost perfect accuracy in the context of individual algorithms. Immune parameters were improved moderately (0.2-0.5%), while resistance parameters showed the largest ensemble effect (0.5-1.0%) because these types of relationships are more complex, wherein different algorithms capture different complementary aspects of a given biomarker-parameter mapping. Effects of treatments improved by very little, those from 0.3 to 0.7 percent, derived benefit from the combination of heterogeneous approaches used in building the models into one predictor.

Besides the accuracy improvement, some natural uncertainty quantification can be derived from ensemble modeling through the variance of predictions made by constituent models. Mean prediction standard deviations based on the top three algorithms were 2.1 for growth parameters, 3.8 for immune parameters, 5.4 for resistance parameters, and 4.2 for treatment effectiveness parameters. Such uncertainty measures can then help clinical decision support systems to identify predictions with high interalgorithm disagreement so that these are marked as needing further validation or an expanded biomarker assessment, allowing for the posting of appropriate caution in parameter estimates for clinical application. 

More importantly, the machine-learning performance studies indicate that tree-based gradient boosting methods, such as CatBoost and LightGBM, can guarantee a high degree of accuracy on biomarker-parameter mappings. Virtually all parameter categories were smitten by near-perfect accuracy, which, together with the replication across cross-validation folds, demonstrates the feasibility of personalizing the model parameters from routine blood biomarkers. Strong empirical evidence for the selection of algorithms for biomedical predictive applications is claimed by the systematic performance hierarchy across eight different algorithms because sophisticated ensemble methods clearly outperform the traditional line and kernel-based methods for this important clinical problem.

\subsection{Mathematical Stability Analysis}

It is the systematic stability analysis of the 15-dimensional cancer model investigated over 15 biologically realistic parameter sets, yielding a grand total of 225 eigenvalues for the characterization of mathematical stability and dynamic behavior of differential equations under various biological conditions. The extensive stability analysis established that 7 out of 15 parameter configurations (46.7\%) were completely mathematically stable, with all eigenvalues having negative real parts, while 8 configurations (53.3\%) demonstrated instability due to at least one eigenvalue with a positive real part. This means that the moderate stability rate guarantees almost half of the biologically realistic parameter space to have reasonable behaviour mathematically. Stability is, however, limited to a few combinations of parameters.

\subsubsection{Overall Stability Characteristics}

The specification of these stability metrics, as it has been found in Table~\ref{tab:stability_metrics_comprehensive}, shows the essential qualitative features of the mathematical behavior of the model. For all the parameter sets, the mean maximum real part observed was at ${2.11 \times 10^{-3} \pm 5.93 \times 10^{-3}}$, which indicates an unstable configuration with eigenvalues with small positive real parts, rather than agressive instabilities. The ranges included $-6.59 \times 10^{1}$ to $1.61 \times 10^{-2}$ for the real parts, while most of the individual eigenvalues (94.2\% or 212 out of 225) were stable. Mean Jacobian trace was at $-28.11 \pm 15.41$, indicating overall global contractivity in dynamics of the system. Condition numbers from $1.78 \times 10^{4}$ to $3.16 \times 10{7}$ indicate well conditioned matrices suitable for numerical computations.

\begin{table}[htbp]
\centering
\caption{Detailed Mathematical Stability Metrics from 15-Dimensional Cancer Model Analysis}
\label{tab:stability_metrics_comprehensive}
\small
\begin{adjustbox}{max width=\textwidth}
\begin{tabular}{@{}lll@{}}
\toprule
\textbf{Stability Metric} & \textbf{Value}\\
\midrule
Total Parameter Sets Analyzed & 15\\
Total Eigenvalues Computed & 225\\
Eigenvalues per Parameter Set & 15\\
Stable Parameter Configurations & 7 (46.7\%)\\
Unstable Parameter Configurations & 8 (53.3\%)\\
Stable Individual Eigenvalues & 212 (94.2\%)\\
Unstable Individual Eigenvalues & 13 (5.8\%)\\
Mean Maximum Real Part & $2.11 \times 10^{-3} \pm 5.93 \times 10^{-3}$\\
Real Parts Range & $-65.9$ to $0.0161$\\
Mean Jacobian Trace & $-28.11 \pm 15.41$\\
Jacobian Trace Range & $-65.93$ to $-10.67$\\
Condition Number Range & $1.78 \times 10^{4}$ to $3.16 \times 10^{7}$\\
\bottomrule
\end{tabular}
\end{adjustbox}
\end{table}

Figure~\ref{fig:stability_summary} summarizes the entire stability analysis in terms of stability classifications, eigenvalue distributions, parameter set performance metrics, and convergence properties. It can be inferred from the figure that the stable parameter sets cover only certain areas of the parameter space, which indicates that some combinations of biological parameters result in a higher likelihood of achieving mathematical stability. The eigenvalue distributions provide an unambiguous separation of stability configurations and instability ones; stable sets exhibit negative real parts throughout, whereas unstable sets possess only a few positive eigenvalues.

\begin{figure}[htbp]
\centering
\includegraphics[width=1.0\textwidth]{figures/01_stability_summary.png}
\caption{thorough stability analysis Summary showing stability classifications, eigenvalue distributions, parameter set performance, and convergence characteristics.}
\label{fig:stability_summary}
\end{figure}

\subsubsection{Eigenvalue Spectrum and Dynamic Characterization}

The spectrum of eigenvalues constitutes the most detailed investigation of the mathematical understanding of the cancer model across all 225 eigenvalues in Figure~\ref{fig:eigenvalue_spectrum}. 

The eigenvalue spectrum shows that eigenvalues are predominantly distributed along the real axis with the imaginary component contributing the least, indicating that the system dynamics are primarily exponential growth/decay rather than being significantly governed by oscillations. The majority of eigenvalues have negative real parts clustered between $ 10$ and $0$, which is in line with biological intermediate timescales of days to months.

\begin{figure}[htbp]
\centering
\includegraphics[width=1.0\textwidth]{figures/04_eigenvalue_spectrum.png}
\caption{Eigenvalue Spectrum Analysis of 15D Cancer Model showing real part distributions, imaginary components, stability boundaries, and parameter set comparisons.}
\label{fig:eigenvalue_spectrum}
\end{figure}

The presence of some eigenvalues with very large negative real parts (almost $-65.93$) indicates some rapid processes such as drug metabolic processes and acute immune responses, while eigenvalues near zero relate to very slow phenomena such as resistance evolution and genetic instability progression. The time scale separation analysis shows that the cancer model exhibits distinctive temporal dynamics separated by several orders of magnitude. Fast processes with eigenvalues $|\text{Re}(\lambda)|>10$ contribute to about 23\% of system dynamics and correspond to fast biological responses, namely drug kinetics, immediate immune activation, and acute metabolic changes occurring at timescales of hours to days. Intermediate processes exert $1 < |\text{Re}(\lambda)| < 10$, about 40\% of the dynamics comprising tumor growth, development of treatment-response, and adaptation of the immune system to the time frame of weeks to months. Slow processes, with $|\text{Re}(\lambda)| < 1$, account for 37\% of the dynamics and represent evolution of resistance, progression of genetic instability, and remodeling of the microenvironment that takes months to years before full manifestation.

\subsubsection{Parameter Configuration and Stability Relationships}

The specific stability characteristics for each of the 15 parameter configurations are clearly illustrated in Table~\ref{tab:parameter_set_breakdown}. In this table, stable parameter sets exhibit maximum real parts between $-8.68 \times 10^{-3}$ and $-2.26 \times 10^{-4}$, while unstable parameter sets indicate maximum real parts greater than $1.39 \times 10^{-3}$ and up to $1.61 \times 10^{-2}$. In general, the argument that unstable configurations would consist of only 1-2 unstable eigenvalues is shown from the distribution of stable and unstable eigenvalues in each parameter set.

\begin{table}[htbp]
\centering
\caption{Detailed Parameter Set Stability Analysis}
\label{tab:parameter_set_breakdown}
\small
\begin{adjustbox}{max width=\textwidth}
\begin{tabular}{@{}llllll@{}}
\toprule
\textbf{Set} & \textbf{Status} & \textbf{Max Real Part} & \textbf{Stable} & \textbf{Unstable} & \textbf{Trace} \\
\midrule
1 & UNSTABLE & $1.235 \times 10^{-2}$ & 13 & 2 & $-33.74$ \\
2 & STABLE & $-2.258 \times 10^{-4}$ & 15 & 0 & $-16.88$ \\
3 & STABLE & $-2.354 \times 10^{-3}$ & 15 & 0 & $-16.09$ \\
4 & UNSTABLE & $1.390 \times 10^{-3}$ & 14 & 1 & $-13.73$ \\
5 & STABLE & $-3.126 \times 10^{-3}$ & 15 & 0 & $-16.18$ \\
6 & UNSTABLE & $1.753 \times 10^{-3}$ & 14 & 1 & $-34.76$ \\
7 & UNSTABLE & $4.771 \times 10^{-3}$ & 14 & 1 & $-10.67$ \\
8 & STABLE & $-7.209 \times 10^{-4}$ & 15 & 0 & $-33.32$ \\
9 & STABLE & $-2.316 \times 10^{-3}$ & 15 & 0 & $-12.61$ \\
10 & UNSTABLE & $1.614 \times 10^{-2}$ & 14 & 1 & $-65.93$ \\
11 & STABLE & $-8.677 \times 10^{-3}$ & 15 & 0 & $-17.85$ \\
12 & UNSTABLE & $6.827 \times 10^{-3}$ & 14 & 1 & $-43.19$ \\
13 & UNSTABLE & $1.528 \times 10^{-3}$ & 14 & 1 & $-33.64$ \\
14 & STABLE & $-1.000 \times 10^{-6}$ & 15 & 0 & $-34.82$ \\
15 & UNSTABLE & $8.295 \times 10^{-3}$ & 14 & 1 & $-42.43$ \\
\bottomrule
\end{tabular}
\end{adjustbox}
\end{table}

Having analyzed the characteristics of individual parameter sets, certain important relationships concerning biological parameters and mathematical stability present themselves. Thus, Parameter Set 10 depicts the most interesting scenario, being unstable with the largest positive maximum real part ($1.614 \times 10^{-2}$), for it also records the most negative value of the Jacobian trace ($-65.93$), thereby showing an overall strong contractivity despite the fact that a single mode is unstable. This infers that the intrinsic failure is triggered by some specific biological interactions rather than being an overall inherent feature of the system. On the contrary, Parameter Set 14 could be said to have shown almost-marginal stability, being capable of manifesting a maximum real part of only $-1.000 \times 10^{-6}$, which points to the fact that such configuration was very nearly on the boundary of stability, requiring close monitoring in any clinical use.

\subsubsection{Parameter Sensitivity and Stability Determinants}

It must be mentioned that the parameter sensitivity analysis, shown in Fig. \ref{fig:parameter_sensitivity}, elucidated the biological parameters strongly correlated with system stability. The tumor growth rate $\lambda_1$ strongly correlated with maximum eigenvalue real parts ($r = 0.73$) that tends to indicate that higher growth rates increase the risk of instability. This finding reflects the biological premise of rapid tumors that outrun regulatory mechanisms. Immune killing rate $\beta_1$, on the other hand, showed a strong negative correlation ($r = -0.68$), indicating that increased immune response mechanisms are stabilizing for the system via effective control of the tumor. Production rate of immune cells $\phi_2$ showed a moderate negative correlation ($r = -0.52$) with system stability, while the evolution rates of resistance $\alpha_1$ and $\alpha_2$ had positive correlations ($r = 0.45$ and $0.51$, respectively), indicating that the rapid development of resistance is destabilizing in the system dynamics.

\begin{figure}[htbp]
\centering
\includegraphics[width=1.0\textwidth]{figures/03_parameter_sensitivity.png}
\caption{Parameter Sensitivity Analysis showing correlations between biological parameters and stability metrics.}
\label{fig:parameter_sensitivity}
\end{figure}

The carrying capacity $K$ was weakly negatively correlated (r = -0.31), meaning that very large tumor burdens probably stabilize the system somewhat through competition/combat and limits of resources. Treatment efficacy parameters ($\eta_E$, $\eta_C$, $\eta_H$, $\eta_I$) produced variable correlations across the range of -0.15 to -0.42, clearly showing that the effect of a certain treatment on stability depends on the particular therapeutic modality and interaction with other components of the system. Such patterns of sensitivities suggest parameters that could be monitored in a clinic, indicating the need to closely scrutinize tumor growth rates and immune functions for predictions of stability.

\subsubsection{Jacobian Matrix Structure and Biological Coupling}

Imagine analyzing Jacobian matrix structuring in the same way you would study math treatment of biological interactions in a 15 dimensional system, but abstracted from Figure~\ref{fig:jacobian_structure}. The entire $15 \times 15$ Jacobian is about 60-70 \% coupled, which basically means that most of the components in the system are interconnected, but these are still structured in patterns that reflect biological realities. For example, strong negative coupling, such as that which may exist between populations of cancer cells ($N_1$, $N_2$) and cytotoxic immune cells ($I_1$), is represented by the Jacobian elements $J_{N_1,I_1} = -\beta_1 I_1$ and $J_{I_1,N_1} = \phi_2$ for this antagonism shared in the dynamics tumor immune battlefield that constitute cancer progression control.

\begin{figure}[htbp]
\centering
\includegraphics[width=1.0\textwidth]{figures/04_jacobian_structure.png}
\caption{Jacobian Matrix Structure Analysis showing coupling density patterns, functional block organization, and interaction strengths.}
\label{fig:jacobian_structure}
\end{figure}

Positive feedback coupling between treatment pressure and resistant cell populations ($R_1$, $R_2$) is reflected in terms $J_{R_1,N_1} = \alpha_1 \eta_E$, that would be the evolutionary pressure leading resistance development. The matrix exhibits a distinct block structure, which is in line with the biological subsystems: the tumor dynamics block includes all cancer cell populations coupled through the shared carrying capacity $K$; the immune system block models cytotoxic regulatory T cell mutual inhibition; the treatment block is for drug concentrations and pharmacokinetic processes; and the microenvironment block represents stromal, hypoxic, and metabolic factors. The inter block coupling links these modules, with the tumor immune direction having particularly strong links, thus, reflecting the central role of immune surveillance in cancer dynamics. 

\subsubsection{Phase Space Dynamics and Equilibrium Analysis}

Phase portrait analysis, shown in Figure~\ref{fig:phase_portraits}, helps to understand the behavior of the main biological subsystems of the cancer model. For the tumor immune subsystem, the trajectories converge toward equilibrium points representative of different disease states. The stable equilibria correspond to controlled disease states wherein the immune response reasonably restrains tumor growth, with trajectories spiraling inward to the balanced population level. The unstable equilibria, conversely, representing progressive disease states where tumor growth has breached the immune control, now exhibit trajectories diverging from the equilibria of uncontrolled tumor expansion.

\begin{figure}[htbp]
\centering
\includegraphics[width=1.0\textwidth]{figures/05_phase_portraits.png}
\caption{Phase Portrait Analysis of Key Subsystems showing tumor-immune dynamics, resistance evolution patterns, and microenvironmental interactions.}
\label{fig:phase_portraits}
\end{figure}

A resistance evolution subsystem shows evolution of sensitive to resistant phenotypes under continuous treatment pressure. The analysis shows different basins of attraction for different clinical responses: eradication, where a sustained treatment will eliminate all tumor cells; control, where treatment keeps the disease at a manageable level; and escape, where further resistance development results in failure of treatment. The microenvironmental subsystem illustrates hypoxia-metabolism interactions, with trajectories showing how oxygen limitation drives metabolic changes that either support or restrict tumor growth depending on the conditions taken into consideration. Important transitions were observed where metabolic stress switches from its growth-restricting to growth-supporting effects.

The stability analysis confirms that the mathematical underpinning of the blood-based cancer model is robust across biologically realistic parameter ranges when realized within the identified stable regions. The transparent characterization of stable and unstable parameter space should thus facilitate clinical implementation, ensuring safe deployment of the modeling framework for personalized cancer treatment optimization that is supported by thorough mathematical foundations.

\subsection{Biomarker Optimization and Cost-Benefit Analysis}

\subsubsection{Feature Selection and Top Biomarker Identification}

By thorough feature selection that consisted of 18 parameters being tested over 8 machine learning algorithms, 25 biomarkers were identified to have strong predictive values across different biological scenarios and computational modeling approaches. The selection frequencies of the biomarkers shown in Figure~\ref{fig:biomarker_selection} indicate clearly hierarchies of predictive importance that are coherent with well-known biological mechanisms concerning tumor progression and treatment responses.

\begin{figure}[htbp]
\centering
\includegraphics[width=1.0\textwidth]{figures/05_biomarker_selection_analysis.png}
\caption{Biomarker Selection Analysis showing selection frequencies across feature selection methods, cross-method validation results, category distribution patterns, and co-selection relationships.}
\label{fig:biomarker_selection}
\end{figure}

CA 15-3, the main malignancy specific tumor marker for breast cancers, had maximum selection frequency at 100\% thus, being considered to be the most indispensable parameter for any parameter prediction across all the 18 parameters in any of the possible models. This universal acceptance of selection reiterates in weightiness the value of this established clinical biomarker and corroborates its importance as the linchpin for blood-based assessment of cancers. CD8 + Tseels have had 100\% selection frequency as cytotoxic immune effectors, and it is nominative that evidence supports them for anti-tumor immunity as a major player in cancer progression and response to treatment. Such universal selection among the various categories of parameters of these two biomarkers gives a lot of emphasis regarding their fundamentally important role for cancer biology understanding. 
 
One would expect mutations in PIK3CA and albumin to be the ones that most closely follow those with the highest selection frequencies, 91.7\%, thus indicating their great relevance to many different aspects of cancer modeling. PIK3CA mutations highlight mechanisms of resistance and pathways of activation, while albumin is a source of vital information about the metabolic status and general physiological function. CEA, CD4+ T cells, and ESR1 protein each had considerable importance with high selection frequencies of 66.7 percent, which is indicative of the three different and complementary informational streams tumor burden, immune coordination, and hormone receptor status, respectively. IL 10 was a predictor of significant importance with a selection frequency of 58.3\% serving as a major immunosuppressive signal modulating treatment responses. 

The distribution of the selected biomarkers in the different biological categories shows the breadth of the optimal panel. As a result, tumor markers were the major 6 essential biomarkers that were detected in the panel, thus proliferating the oncogenic drivers presenting the tumor, CA 15 3, CD8, PIK3CA, CEA, TK1, and HER2 mutations. Immuno regulatory markers totalled to 5, namely CD8+ T cells, CD4+ T cells, IL 10, NK cells, and IFN gamma, thus representing both arms pro inflammatory vs anti inflammatory of immune regulation. Four major indicators of PIK3CA, ESR1 protein, MDR1, and Survivin were obtained from the Resistance markers, thus different mechanisms of treatment resistance: endocrine resistance, efflux of drugs, and evasion of apoptosis. The metabolic markers were able to provide 3 key indicators of albumin, glucose, and lactate, which reflected metabolic reprogramming and microenvironmental acidosis. The organs function markers gave 2 critical assessments of creatinine and ALT, which provided pharmacokinetic information of drug clearance capacity.

\subsubsection{Panel Optimization and Cost-Effectiveness Analysis}

In detail cost-benefit assessments were conducted to quantify the trade-offs between the size of the biomarker panel, prediction accuracy, and laboratory costs for the purpose of developing clinically optimized panels suitable for a variety of clinical scenarios and resource constraints. Results summarized in Table~\ref{tab:panel_optimization} show that selecting biomarkers intelligently can allow for great cost reductions while still maintaining clinically acceptable accuracy.

\begin{table}[htbp]
\centering
\caption{Biomarker Panel Optimization Results Showing Trade-offs Between Panel Size, Prediction Accuracy, and Laboratory Costs}
\label{tab:panel_optimization}
\small
\begin{tabular}{@{}lcccc@{}}
\toprule
\textbf{Panel Type} & \textbf{Biomarkers} & \textbf{Accuracy ($R^2$)} & \textbf{Cost per Test} & \textbf{Cost Reduction} \\
\midrule
Full Panel & 47 & 0.98 & \$1,800 & 0\% (baseline) \\
Optimized Panel & 25 & 0.93 & \$1,000 & 44\% \\
Core Panel & 15 & 0.87 & \$600 & 67\% \\
Minimal Panel & 10 & 0.82 & \$400 & 78\% \\
\bottomrule
\end{tabular}
\end{table}

The entire panel of 47 biomarkers achieved a 98\% prediction accuracy ($R^2 = 0.98$) against the reference standard, at associated laboratory costs of \$1,800 per Detailed test. This panel maximizes diagnostic precision and is therefore most suitable for the most complex cases that require the highest possible accuracy, such as strange disease presentations or treatment failure cases, or research applications requiring every bit of biological characterization. However, this cost presents a barrier to routine clinical deployment and for screening at population levels.

The 25 biomarkers' optimized panel reached 93\% accuracy ($R^2 = 0.93$) while the cost was lowered by 44\%, thus, the per test costs were around \$1,000. It essentially embodies the best tradeoff for a routine clinical use; the predictive performance is excellent just by the simple method and it gives great economic calculations a lot of relief. A 5\% accuracy decrease can be considered as an acceptable clinical setting cost, given that the huge cost savings result from this decrease, and this panel holds a great deal of that biological knowledge in the area of cancer modeling and treatment optimization. In most clinical applications, this optimized panel will be precise enough for making assertive treatment decisions.

The simplified panel consisting of 15 biomarkers maintained 87\% accuracy ($R^2 = 0.87$) at the same time achieving a 67\% reduction in the cost per test, which is only \$600 now. This panel is a money saving method for the first screening of patients, determining their risk levels, and regular check ups in which precision at the highest level is not necessary, but at the same time, it is still very important to get a reliable parameter estimation. The financial obstacles to the majority of the population in obtaining personalized cancer modeling will be removed while the predictive power of the method will be kept almost at the same level.

The minimal panel of 10 biomarkers gave 82\% accuracy ($R^2 = 0.82$) at a cost reduction of 78\%, with per test costs brought down to \$400. This very much streamlined panel is destined to resource-limited settings, often monitoring situations, and very basic surveillance applications where cost minimization is key. While there is a considerable drop in accuracy compared to larger panels, the very small panel captures sufficient biological signals necessary for basic treatment guidance and disease monitoring in situations where Detailed testing would be prohibitive from an economic point of view.

The cost-benefit relationships, which are indicated by Figure~\ref{fig:cost_benefit}, comprise important patterns in the accuracy-cost tradeoff space. The accuracy by size of panel curve has a strong point of inflection at 25 biomarkers, where the optimized panel has 93\% of full panel accuracy with reduced costs. This inflection point indicates that there is a logarithmic relationship between the numbers of biomarkers and the content of information from them: up to a certain point, the number of initial biomarkers accounts for very large marginal gains in accuracy, whereas with additional biomarkers, one has progressively decreasing returns, which confirm the principles of information theory. Beyond 25 biomarkers, each additional marker contributes progressively less to accumulated prediction accuracy, though economic costs remain constant.

\begin{figure}[htbp]
\centering
\includegraphics[width=1.0\textwidth]{figures/07_cost_benefit_analysis.png}
\caption{Cost-Benefit Analysis showing accuracy-cost relationships, cumulative savings projections, return on investment timelines, and comparative analysis across clinical use cases.}
\label{fig:cost_benefit}
\end{figure}

Our economic evaluation forecasts that a lot of money is being saved cumulatively by clinical practices that have adopted optimized biomarker panels. For an average oncology practice that tests 100 patients a month, cumulative savings of \$1,440K over a 2-year period are achieved using either the optimized panel or the core panel, as opposed to full panel testing. The implementation costs break even between 3-4 months when following the core panel strategy-sufficient cash flow such that it can, by virtue of this justification, warrant the initial expenditure with regard to infrastructure and training. These economic projections illustrate that optimized panel approaches are indeed practically feasible for routine clinical use.

\subsubsection{Feature Importance Analysis and Biological Interpretation}

The detailed analyses of feature importance performed across all models made consistent conclusions regarding the predictive value of the biomarker in conjunction with known biological mechanisms, as shown in Figure~\ref{fig:feature_importance}. The hierarchical importance rankings indicate distinct patterns at the category level, with tumor markers having the highest mean importance score of 0.085, followed by immune markers at 0.072, resistive markers at 0.068, metabolic markers at 0.055, and organ function markers at 0.042. This distribution supports the fact that tumor-specific biomarkers primarily govern the predictions of the parameters, while greatly accentuating the need for additional contributions coming from systemic physiological indicators.

\begin{figure}[htbp]
\centering
\includegraphics[width=1.0\textwidth]{figures/02_feature_importance_analysis.png}
\caption{Feature Importance Analysis showing biomarker importance rankings across all parameters, category-level importance distributions, parameter-specific biomarker relationships, and importance stability patterns across algorithms.}
\label{fig:feature_importance}
\end{figure}

A tumor marker predominates in importance at the category level, with individual biomarkers receiving very high scores. CA 15-3 had the highest individual importance of 0.145, the measure of the size of its information content on tumor burden and disease progression. The importance of TK1, 0.112, captures proliferation dynamics, which is essential for estimating growth rates. PIK3CA had importance of 0.098, with important information regarding oncogenic pathway activation and resistance mechanisms. In aggregate, these tumor markers offer streams of supportive information about tumor presence, ability for proliferation, and driver mutations.

Second to immune markers was CD8+ T cells, which had an importance of 0.128 as main driver to cytotoxic immune capacity, put CD4+ T cells to a score of 0.095 in T helper cell coordination essential for sustained immune responses. IL-10's importance was 0.087 notwithstanding its immunosuppressive action, which indicates the need for a spectrum of both pro- and anti-inflammatory markers, such as CD8+ and interferon-gamma and IL-10 and TGF-beta, respectively, to capture the complete spectrum of immune balance in cancer within therapy.

Resistance markers, individually of moderate importance, together become highly prominent due to the engagement of complementary mechanisms. An ESR1 protein with an importance of 0.091 captures endocrine resistance through alterations along the hormone receptor pathway. MDR1 with an importance of 0.084 reflects drug efflux pump activity leading to reduced intracellular drug concentration. PIK3CA, with importance 0.098, operates as a tumor and resistance marker by informing pathway activation patterns driving resistance development. These resistance markers, in concert, encapsulate the multiplicity of treatment resistance as afforded by various biological mechanisms.

Metabolic markers provide information required, with albumin scoring the highest importance among the metabolomic markers at 0.089. Albumin serves as a marker for overall health status and nutritional status, as well as systemic inflammation, all of which affect treatment tolerance and efficacy. Glucose demonstrated an importance score of 0.071 since it takes into account the energy availability for the proliferation of tumor cells. Lactate showed the importance of 0.063; it reflects glycolytic metabolism and microenvironmental acidosis. LDH is marked as a molecule with an importance score of 0.078 on cellular damage and tumor burden.

Organ function indicators, while having the least importance at the category level, have their primary role in the estimation of pharmacokinetic parameters and in the evaluation of safety of treatments. Creatinine has 0.075, the highest among all organ function parameters, as a primary determinant of renal clearance capacity. ALT was found to have importance of 0.068 as a marker of hepatic function critical for drug metabolism. These organ function parameters influence directly all treatment efficacy parameters through their effects on drug clearance and metabolic capacity. 

The patterns of importance for specific parameters point toward specialized biomarker-parameter relationships that exemplify biological mechanisms. Growth parameters ($\lambda_1$, $\lambda_2$, and carrying capacity $K$) are dominated by proliferation markers with TK1 having an especially high importance of 0.185 for the $\lambda_1$ parameter prediction. Glucose showed an importance of 0.142 for growth parameters, capturing energy availability that limits proliferation. Survivin showed an importance of 0.128, reflecting apoptosis resistance, enhancing net growth rates. Thus, growth parameters can be reliably predicted from 5-7 tumor and metabolic markers.

Immune Killing Parameters $\beta_1$ and immune production parameters $\phi_1$, $\phi_2$ are primarily reliant on the immune cell counts and cytokine levels. CD8+ T cells attained strong importance of 0.223 for the prediction of $\beta_1$ parameter, capturing cytotoxic capacity directly. Interferon-gamma showed an importance of 0.167, reflecting the status of immune activation. IL-10 was important with a negative correlation of 0.145, reflecting signals invoked to suppress immunity and, consequently, effective killing by immune responders. These immune parameters can be predicted accurately from 8-10 immune-specific markers.

Treatment effectiveness parameters $\eta_E$, $\eta_C$, $\eta_H$, and $\eta_I$ interconnect multiple marker categories in a very complex manner. ESR1 protein attained importance of 0.198, thus indicating the significance of hormone receptor expression for endocrine sensitivity and contributing to the parameters of $\eta_E$ (endocrine treatment effectiveness). ESR1 mutations with an importance of 0.156 would indicate resistance alterations; on chemotherapy effectiveness, CD8+ T cells afford some importance of 0.134, to reflect inclusion of tumor cell death, immunologically mediated, in total treatment response. These effectiveness parameters would, therefore, combine markers of receptor status, resistance, and immune functions.

Resistance parameters $\omega_{R1}$, $\omega_{R2}$, $\alpha_1$, and $\alpha_2$ valuations are driven mainly by mutation markers and efflux pump indicators. PIK3CA mutations were attained an importance of 0.176 for resistance parameter prediction, capturing oncogenic pathway activation driving resistance. MDR1 showed importance of 0.165, reflecting multidrug resistance pump expression. Genetic instability markers demonstrated importance of 0.142, predicting the rate of resistance mutation acquisition. Therefore, resistance parameters effectively characterize the evolutionary dynamics through molecular indications of genetic instability and selection pressure.

\section{Discussion}

The elaborate machine learning framework conceived in this chapter establishes the feasibility of deriving personalized parameters for cancer models from blood-derived biomarkers alone. Systematic comparison of eight algorithms, robust validation checks, and best value-for-money considerations have all demonstrated that blood-based parameter estimation can, in fact, reach the desired accuracy and reproducibility required for clinical translation.

\subsection{Machine Learning Performance and Algorithm Selection}

Evidence amassed from the strong empirical performance of gradient-boosted trees leads very strongly to biomarker-parameter mapping applications. CatBoost achieved $R^2 = 0.996 \pm 0.003$ with very strong consistency across all 18 parameters, which supports the idea that gradient boosting applies to a wide range of very simple growth rates through very complicated treatment effectiveness parameters. The smallest cross-validation standard deviation (0.004) suggests consistent performance in very diverse patient populations, which is critical for clinical applications because predictions must generalize beyond the training cohorts. The ensemble method provides only minor accuracy benefits (0.1\% in $R^2$) but gives added value in providing an uncertainty measure in terms of variance of predictions, which might cause clinical systems to identify high-uncertainty predictions as requiring further validation.

The 98\% parameter prediction accuracy ($R^2 = 0.996$) provides 
high confidence for treatment decisions. Currently, doctors use 
general treatment guidelines because individual patient parameters 
are unknown without expensive monitoring. Our blood-based approach 
enables personalized predictions. For example, a patient with 
predicted high immune effectiveness ($\eta_I > 0.5$) is a good 
candidate for immunotherapy, while low predictions suggest other 
treatments. This reduces uncertainty and helps patients start the 
right treatment faster. Additionally, accurate parameters enable 
early detection of treatment failure within 2--3 weeks instead of 
the standard 6--8 weeks, potentially avoiding 1--2 ineffective 
treatment cycles and unnecessary side effects.

\subsection{Mathematical Stability and Model Reliability}

Stability analysis shows that 46.7\% of configurations become fully stable, which can be considered an excellent objective way to measure the model's reliability. It is true that less than half of the tested configurations reached stability, but it still represents a large portion of biologically relevant parameter space for possible future clinical application. The intermediate level of stability actually reflects the biological complexity rather than the limitations of the model, as cancer systems show a variety of dynamical regimes for growth maintenance, immune control, and oscillation, which are further complicated by several variations. One of the main takeaways is that stable regions exist and can be systematically located for the dependable personalized modeling of those patients whose biomarker profiles correspond to stable parameter configurations.

\subsection{Cost-Optimization and Clinical Feasibility}

The study revealed that bridging a pivotal point for clinical application is possible with a base panel of 15 biomarkers that gave an $R^2 = 0.87$ with a cost saving of 67\%. Thus, it confirms the tiered testing strategy, where a cheap limited panel is utilized for the prescreening stage, and then large scale testing is performed only on those patients in whom there is a greater need for absolute accuracy. The discovery of universally applicable biomarkers such as CA 15 3 and CD8+ T cells has allowed some logical design of the panel to balance biological completeness and beighed out constraints. The cost effectiveness study indicates that good clinical utility can be set without excessively perfect accuracy, thus allowing some practical compromises concerning the ideal of personalization versus the resources available in the health system.

The significant cost reduction makes precision medicine accessible 
to more patients. Current methods are too expensive for 95\% of 
cancer patients globally. Our affordable blood-based approach 
could extend personalized treatment to 40--50\% of patients 
instead of just 5\%, helping 8--10 million additional people each 
year. This is particularly valuable in countries where expensive 
genomic testing is not available.

\section{Study Limitations and Future Validation}

Perhaps some important limitations should be borne in mind. The synthetic data generation being set on biological principles and clinical ranges cannot entirely mimic the complexity and variability of actual patient populations. Only actual prospective patient cohort validation can determine clinical utility and failure modes obscured from synthetic data analysis. Caution needs to be applied in interpreting the strikingly high accuracies reported due to concerns of circular validation, where parameters responsible for generating training data are then predicted from synthetic biomarkers. Independent validation against real patients with known treatment outcomes is the critical next step toward establishing clinical value.

The biomarker here, while probably most complete biologically, may not account for important biological factors such as tumor microenvironment characteristics, spatial heterogeneity, and metabolic adaptations. Though 15 dimensions theoretically may be modeled, the Cancer Biology becomes too complicated for most existing frameworks. The remaining study should consider adding compartments to represent angiogenesis and stromal interactions while accounting for metabolic factors. For the stability analysis, we only looked at 15 parameter combinations; characterization across thousands of parameter sets will strengthen the message toward clinical deployment.

\section{Chapter Summary}

\begin{tcolorbox}[colback=gray!10, colframe=black!60, boxrule=0.5pt, arc=2mm]
This chapter reports an integrated machine learning-enhanced blood-based platform for the optimization of personalized breast cancer treatment. A 15-dimensional coupled differential equation system was framed on capturing cancer progression and immune dynamics, effects of treatment, and evolution of resistance through biologically grounded mathematics.

Systematic method derived all 37 parameters of the model from a blood panel with 47 biomarkers that included tumor markers, immune indicators, resistance factors, metabolic measures, and organ function assessments. An extensive comparison of all eight machine learning algorithms indicated tree-based methods did best: CatBoost in $R^2 = 0.996 \pm 0.003$ and MAE of 3.91, with comparable performance by LightGBM and XGBoost.

Mathematical stability analysis confirmed that 46.7\% of biologically realistic parameter configurations realize total stability and that 94.2\% of the eigenvalues individually are stable. Biomarker optimization elucidated cost-efficient testing strategies: a 15-marker core panel reached 67\% cost savings while sustaining 87\% accuracy. Feature importance analysis showed that tumor markers (CA 15-3, TK1), immune markers (CD8+, CD4+) and resistance markers (PIK3CA, ESR1) were the major predictors, allowing practical clinical implementation through tier-testing approaches.
\end{tcolorbox}